%\begin{color}{red}
%\begin{itemize}
%  \item Projected content \cite{simons2010projects,tonhauser2013toward}; 
%    Conventional Implicatures \cite{potts2005logic}
%  \item Presupposition as anaphora \cite{sandt1992presupposition,
%    geurts1999presuppositions}; Discourse Representation Theory
%    \cite{kamp1993discourse}
%  \item Projective DRT \cite{venhuizen2013iwcs-short}: solution
%  \item Implementation
%\end{itemize}
%\end{color}

%%%% Introduction

%The meaning of an utterance is a complex combination of different types of
%content, including references to information already present in the common
%ground and novel contributions to it. The information status of some piece
%of content is signalled through the information structure of the utterance,
%by means of a specific choice of words or a linguistic construction. For
%example, the use of a definite determiner signals reference to an already
%introduced discourse entity, whereas indefinites signal the introduction of
%new entities. When modeling the semantic contribution of some utterance, it
%is thus important to reflect its information structure. 

% A straightforward distinction can be made between {\it old} and {\it new
% information}. Another way to distinguish different types of meaning is on
% the basis of the {\it Question Under Discussion} (QUD)
% \cite[cf.][]{roberts1996information-short}; content that addresses the QUD
% can be called {\it at-issue}, whereas backgrounded, secondary information
% is {\it not at-issue}. Simons et al. \cite{simons2010projects} showed
% that at-issueness has important implications for the semantic properties
% of linguistic content.

%Despite the semantic significance of CI content, it has mainly been
%neglected in formal semantic frameworks. In this paper, we propose a new
%formal treatment of CIs, which highlights their informative contribution.
%As a formal framework we use Projective DRT \cite{venhuizen2013iwcs-short},
%a recently proposed extension of Discourse Representation Theory
%\cite{kamp1993discourse} that deals with projected content by means of
%variable binding. We show that the treatment of CIs proposed in
%\cite{venhuizen2013iwcs-short} is unsatisfactory and propose an alternative.
%Finally, we present an open-source Haskell implementation of PDRT that is
%used to evaluate the treatment of CIs and other projected content.


%%%% About CIs
%new information/ accommodation as a default? / anchor/ speaker-orientedness

In contrast to
presuppositions, CIs introduce novel information, which is signalled to be
off-issue by means of subordinated structures, such as appositives, as in
``John\underline{, a linguist,} was not at the party''.

% CI: focus on subordinate clauses %(exclamatives are different? cf
% \cite{geurts2007fucking})

\paragraph{CIs add new information about their anchor.} It is
well-established that CIs differ from presuppositions in that they offer
novel information to the discourse context. The most paradigmatic examples
of CI triggers are supplemental clauses, such as appositives and
non-restrictive relative clauses, shown in \Next.  Here, the relative clause
is anchored to a definite description (``the girl'') that picks out
a referent embedded in an epistemic clause (introduced by ``believes'').
This illustrates observation (i) from above, that CIs only project as far as
their anchor. 
 %(also, CIs allow embedded presuppositions)

\ex. John believes he saw a girl in the park. The girl, who was feeding the
  ducks, smiled.

The second observation, that CIs generally attach to projected content, is
related to the strong correspondence between CIs and discourse anaphora
(a.k.a.  e-type pronouns) \citep{nouwen2007appositives-short}, illustrated
in \Next. Both \Next[a] and \Next[b] are only felicitous if the indefinite
anchor (``a Dutch boxer'') is interpreted specifically. On the other hand,
the status of the information contributed by the supplement in \Next[a] differs
from the contribution of the second sentence in \Next[b] in terms of
at-issueness; while the former is signalled to be off-issue, the latter is
not.

\ex. \a. A Dutch boxer, who is famous, took part in the event.
\b. A Dutch boxer took part in the event. He is famous.
  \hfill(adapted from \citep{nouwen2007appositives-short})

These examples motivate an analysis of CIs that centers around the
dependence to their anchor.


%%% Formalization

%The relation between the introduction and
%interpretation site of projected content is a crucial part of the
%information structure of a discourse, as it reflects the incremental process
%of information assembly.
%%%
%Projective DRT takes the correspondence between anaphora and projection
%a step further by treating projection as variable binding; this way,
%projected material can stay `in situ' while still being discernible from
%asserted material.  Each context introduces a \emph{label} (shown left of
%each box in the examples below) that can bind the \emph{pointers} (indicated
%with `$\gets$') associated with the discourse referents and conditions,
%which indicate where some content is interpreted. Intuitively.. *pointers as
%contexts/files*.
%%%
%\Next[b] shows the new analysis of
%conventional implicatures as projected content that is not referential. The
%CI triggered by the appositive construction projects along with its
%\emph{anchor}--the presupposition triggered by ``Mary''--thus providing new
%projected information without introducing a new projected context.

\begin{flushleft}
\begin{minipage}{\linewidth}
\begin{multicols}{2}
  \ex. 
  %\a. Bill doesn't love Sue.\\
  % \hspace*{-1cm}{\small \pdrs{1}{$3\gets$x}{
  %   $3\gets$Bill(x)\\ $1\gets\neg$\pdrs{2}{$4\gets$y}{
  % $2\gets$love(x,y)\\ $4\gets$Sue(y)}}
  %{$1\leq 2$~~$1\leq 3$~~$2\leq 3$}
  %}\\
  %\vfill\columnbreak
  a.~~~Mary, a linguist, laughs.\\
  \hspace*{0.5cm}{\small \pdrs{1}{$2\gets$x~~$2\gets$y}{$2\gets$Mary(x)\\ 
    $2\gets$linguist(y)\\ $2\gets$y=x\\ 1$\gets$laugh(x)} 
  }
    %{\small $[3~|~1\gets x,2\gets z~|~1\gets Mary(x),2\gets x=z,2\gets linguist(z),3\gets smiles(x)~|~3\leq 1,3\leq 2,2=1]$}\\
  \vfill\columnbreak
  \hspace{-2cm}b.~~~\textit{Appositive comma}:\\\\
  %\hspace*{-1.5cm}~{\small $\lambda k. \lambda l. \lambda m.(~l~(~k~($\pdrsmap{$3$}{
  %    $2\gets x~~3\gets y$}{
  %    $3\gets y = x$}{
  %    $3=2$}$))~*~m (x))$
  \hspace*{-1.5cm}~{\small $\lambda k. \lambda x. \lambda l.(~k~(~\lambda y.$\pdrsmap{$3$}{
      $2\gets x~~3\gets y$}{
      $3\gets y = x$}{
      $3=2$}$)~*~l (x))$

  }

\end{multicols}
\end{minipage}\\
\end{flushleft}

%\paragraph{Implementation.}


