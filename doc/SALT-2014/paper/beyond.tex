\section{Beyond supplements}

\subsection{Speaker-orientedness in PDRT}

Following the traditional model-theoretic interpretation of DRT \citep[see,
e.g.]{kamp1981theory,kamp1993discourse,kamp2011discourse}, a Discourse
Representation Structure (DRS) is thought of as a \textit{partial model}
representing the information conveyed by some utterance, which needs to be
embedded in a \textit{total model} to determine its interpretation. This
embedding is done via an \textit{embedding function} from the set of
discourse referents in the universe of the DRS $K$ into elements in the
domain of the total model $M$, in such a way that all conditions of $K$ are
verified in $M$ (with respect to some world $w$).

To implement speaker-orientation, we can assume that each speaker $S$ has
its own model $M_S$ describing the current state of affairs of the world.
These models may or may not coincide with the model representing the
\textit{actual} state of affairs, which is represented by the `Truth Model'
$M_T$. The idea is that the interpretation a PDRS condition may be
restricted to some (subjective) model, meaning that it only needs to hold in
the designated model. Since in PDRT the information about the interpretation
of linguistic content is explicitly available (in the form of
\textit{pointers}), an interpretational restriction with respect to some
model may also be incorporated as part of the pointer, e.g. via a subscript.
This thus means that for some condition with a pointer of the form `$p_S$',
it holds that this condition is only required to hold with respect to the
model of speaker $S$. Thus, when determining the truth of some utterance
containing some subjective content (i.e., with respect to the Truth Model
$M_T$), the subjective content has to be fulfilled only with respect to the
subjective model $M_S$; in case this holds, the entire utterance will come
out true (provided, of course, that all other conditions are verified with
respect to $M_T$). Note that this places a constraint on the domains of
$M_S$ and $M_T$ insofar that they must overlap with respect to the
referent(s) mentioned in the subjective condition.

\subsection{Other types of CIs}
\subsubsection{Expressives}

We focus on supplemental CIs since these are the most common and
well-studied type of CI.  However, a case can be made for a similar account
of expressives \citep[including expressive attributive adjectives, epithets,
honorifics, and tense-variations such as the German `Konjunktiv I';
cf.][]{potts2005logic}. The status of expressives has been open to some
debate \citep[...]{potts2004japanese,geurts2007fucking}.  But like
supplemental CIs, expressives in general require some antecedent (or:
anchor) to express some opinion about.
%%examples (see, e.g. Amaral et al. p. 711)
However,
the notion of speaker-orientedness is especially important in this context,
so a framework is required that can incorporate speaker attitudes.
[...]
%Q: Do expressives require a projecting anchor?

\noindent\parbox[b]{\textwidth}{\paragraph{Expressive attributive adjectives}
\ex. Sue's dog is \underline{fucking} mean.\\
a.~~~\pdrs{1}{$2\gets$x~~$3\gets$y}{$2\gets$Sue(x)\\ $3\gets$dog(y)\\
  $3\gets$of(y,x)\\ $1\gets$mean(y)\\ $4_S\gets$fucking\_mean(y)}{
  $1<2$~~$1<3$~~$1<4$~~$4=3$}
~~~b.~~~\pdrs{1}{$2\gets$x~~$3\gets$y~~$1\gets$e}{$2\gets$Sue(x)\\ $3\gets$dog(y)\\
  $3\gets$of(y,x)\\ $1\gets$mean(e)\\ $1\gets$Topic(e,y)\\
  $1_S\gets$fucking(e)}{$1<2$~~$1<3$}

}

\noindent In PDRS \Last[a], the anchor of the CI is ``Sue's dog'', which
projects.  However, the representation is not completely satisfactory, since
the adjective ``mean'' is duplicated: it occurs once as part of the asserted
content, and once as part of the expressive CI content. Moreover, the
contraction of ``fucking'' and ``mean'' serving as one predicate goes
against the compositional idea in (P)DRT. In \Last[b], this tension is
resolved by using a neo-Davidsonian event-structure for the adjectival
construction; by letting the adjective ``mean'' introduce an event, which is
related to the subject using VerbNet roles \citep{kipper2008large}, the
expressive adjective ``fucking'' can manipulate the event itself, instead of
the subject. This provides a more intuitive distinction between the asserted
content and the expressive content of this utterance. However, the direct
correspondence to supplemental CIs, which require a projecting anchor, seems
to be obfuscated with this representation.

\noindent\parbox[b]{\textwidth}{\paragraph{Epithets}
\ex. Every democrat advocating a proposal for reform says 
  \underline{the stupid thing} is worthwhile.\\
a.~~\pdrs{1}{}{$1\gets$\pdrs{2}{$2\gets$x~~$2\gets$y}{$2\gets$Democrat(x)\\ 
  $2\gets$proposal\_for\_reform(y)\\ $2\gets$advocate(x,y)}{}$\Rightarrow
  $\pdrs{3}{$3\gets$p}{$3\gets$says(x,p)\\ 
  $3\gets$p:\pdrs{4}{$4,5_S\gets$z}{$5_S\gets$stupid\_thing(z)\\ $4,5_S\gets$z=y\\
  $4\gets$worthwhile(z)}{$4<5$~~$5=2$}}{$3<2$}}{}
b.~~\pdrs{1}{}{$1\gets$\pdrs{2}{$2\gets$x~~$5\gets$y}{$2\gets$Democrat(x)\\ 
  $5\gets$proposal\_for\_reform(y)\\ $2\gets$advocate(x,y)}{
  $2<5$}$\Rightarrow$\pdrs{3}{$3\gets$p}{$3\gets$says(x,p)\\ 
  $3\gets$p:\pdrs{4}{}{$6_S\gets$stupid\_thing(y)\\ 
  $4\gets$worthwhile(y)}{$4<6$~~$6=5$}}{$3<2$}}{}

}

\noindent The epithet ``the stupid thing'' is interpreted in \Last[a] as
contributing both an asserted, anaphoric referent for the adjective
``worthwhile'', and a subjective condition on this referent. By making use
of multiple pointers (which in terms of interpretation comes down to
duplicating the declaration of the referent and the condition), the  dual
role of the expressive noun phrase becomes explicit: it both provides an
antecedent for the asserted statement, and conveys the meaning of the
speaker about this antecedent. Note that the antecedent (``a proposal for
reform'') is itself not a projecting referent, but since the epithet serves
as an anaphor, the object of expression is projected, just as in the case of
supplemental CIs. 
%%explain better
The representation in \Last[b] (where the referent equality is eliminated)
represents the specific reading of the indefinite noun phrase ``a proposal
for reform''. It may be argued that the use of an expressive requires
(presupposes) a projecting anchor, since it only makes sense to express
feelings about some uniquely identifiable object. 

\noindent\parbox[b]{\textwidth}{\paragraph{Honorifics}
\ex. Ame~~ga~~~furi-\underline{mashi}-ta.\\
\textit{rain} \textsc{subj} \textit{fall}-\textsc{hon}-past\\
`It rained.' (\textit{performative honorific})\\
\pdrs{1}{$2\gets$e}{$2\gets$rain(e)\\ $1\gets$fell(e)\\
  $3_S\gets$\textsc{hon}(e)}{$1<2$~~$1<3$~~$3=2$}

}

\noindent The requirement for a projecting anchor is even clearer in the
case of honorifics, since they always serve as definite descriptions that
refer to a specific referent \citep[cf.][]{potts2004japanese}. Therefore,
the `raining event' in \Last is interpreted as a presupposition (referring
to the general concept of rain), so that the performative honorific can be
correctly interpreted by means of projection (again, the referential
equality is resolved).

\noindent\parbox[b]{\textwidth}{\paragraph{German Konjunktiv I} (lack of speaker commitment)
\ex. Sheila behauptet dass sie krank \underline{sei}.\\
\textit{Sheila maintains that she sick~~~be}-\textsc{konj}\\
`Sheila maintains/claims that she is sick.'\\
\pdrs{1}{$4\gets$x~~$1\gets$p}{$4\gets$Sheila(x)\\ $1\gets$behauptet(x,p)\\
  $1\gets$p:\pdrs{2}{}{$2\gets$krank(x)}{$2<4$}\\ 
  $5_S\gets\neg$\pdrs{3}{}{$3\gets$krank(x)}{$3<4$}}{$1<4$~~$1<5$~~$5=4$}

}

\subsubsection{Supplemental Adverbs}


Utterance level modifiers (e.g., ``Frankly, I don't like him'', ``In case
you're hungry, I'm going to the grocery.'') behave differently than other
supplemental CIs \cite[cf.][pp.725-729]{amaral2007review}.


