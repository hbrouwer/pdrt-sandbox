\section{Introduction}\label{Introduction}

Presuppositions have a long history in the semantics literature, in
particular focusing on explaining their `projection behaviour'; what
distinguishes presuppositions from other semantic content is their
indifference to semantic operators like negation, implication, but also
questions, which has been referred to as `projecting' their content to
outside the syntactic scope of embedding operators.  Presuppositions have
generally been considered the most paradigmatic cases of projection
phenomena, but since Potts' redefinition of the class of conventional
implicatures \citep[CIs;][]{potts2003logic,potts2005logic}, the interest has
shifted toward the broader class of phenomena that exhibit the property of
projection \citep[see, e.g.,][]{simons2010projects}.

%XXX examples of presuppositions and CIs

%XXX something about previous approaches (delgobbo, potts, amaral et al,
%nouwen, schlenker, and van der Sandt)

Presuppositions and conventional implicatures share the property of
projection but exhibit clear differences in terms of information structure;
while presuppositions signal established, or \textit{old} information (from
the speaker's perspective), conventional implicatures signal \textit{novel}
information, in the sense that by using a conventional implicature the
speaker communicates to the hearer that he is introducing some additional,
not \textit{at-issue} information that may be new to the hearer.  The
challenge for a formal account of projection phenomena therefore is to
describe the projection behaviour of presuppositions and conventional
implicatures in a unified way, while appreciating their difference in
information structure.

In this paper, we describe such a unified account of projection phenomena,
using the framework of Projective Discourse Representation Theory
\citep[PDRT;][]{venhuizen2013iwcs}. PDRT is an extension of Discourse
Representation Theory \citep[DRT;][]{kamp1981theory,kamp1993discourse},
a wide coverage semantic formalism that has been shown to be able to account
for a variety of phenomena, including (discourse) anaphora and tense
\citep{kamp1981theory}, quantification and plurality
\citep{kamp1993discourse}, attitude reports
\citep{asher1986belief,asher1989belief,zeevat1996neoclassical,maier2009presupposing}
discourse structure \citep{asher2003logics}, and presupposition
\citep{sandt1992presupposition,krahmer1998presupposition,geurts1999presuppositions}.
Projective DRT extends standard DRT with an explicit distinction between the
introduction- and interpretation site of semantic content.
%XXX
