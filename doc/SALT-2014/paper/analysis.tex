\section{An analysis of supplemental CIs}

One of the most prominent accounts of presuppositions is the one by
\citet{sandt1992presupposition}, who treats presupposition projection as
anaphora resolution.
%%explain more.

CIs share their projection behaviour with other projection phenomena, so this
requires a unified analysis of their behaviour.

%\citet{potts2005logic} describes the following properties of CIs:
%
%\begin{itemize}
%  \item CIs tend to project globally (scopelessness)
%  \item CIs are non-cancellable
%  \item CIs are non-restrictive
%  \item Speaker-orientedness
% \end{itemize}

We will argue that all these properties can be described in terms of
a species of indexicality \citep[cf.][]{amaral2007review}, which in the case
of supplemental CIs comes down to indexicality to the projection site of the
anchor.

\subsection{Projection as anchoring}

\subsubsection{CIs are `piggybacking' on their anchor}

* CIs always occur with an anchor\\
* CIs always project as far as their anchor\\
* Similarity to E-type pronouns

\subsubsection{CIs require their anchor to project}

Infelicitous with non-specific indefinites, quantificational constructions

The information structure of CIs (i.e., their subordinatedness) signals
projection; this is why we can distinguish between RRCs and NRRCs --- by
means of the comma in this case. Non-restrictiveness requires projection.
This is because a non-projecting anchor requires the introduction of a new
entity, which by definition lacks a complete description, so any new
information will contribute to the description of the referent and will
therefore be restrictive. On the other hand, in case the anchor projects,
this means that the referent has been established, in the sense that it can
be identified in the common ground, so providing new information does not
affect the choice of referent, so it works non-restrictively.

* example with indefinite anchor -> only specific reading

* example with pronominal anchor -> only felicitous if referent can be
  determined unambiguously.


%Something about generics? and inclusion and attribution as a contrast to
%identification with anchor? \cite[see][]{heringa2012appositional}


\subsection{Projective Discourse Representation Theory}

[short introduction]

\subsection{Projection-anaphoricity}

\subsubsection{Enhancing PDRSs with information structure}

* Adding MAPs to PDRSs. \\
* Projection as \textit{strict} subordination

\subsubsection{Representing the projection behaviour of CIs}

\noindent 
\parbox[b]{\textwidth}{\textbf{Appositive}:
\ex. Sheila believes that Chuck\underline{, a psychopath,} is fit to watch the kids.\\
\pdrs{1}{$3\gets$x~~$1\gets$p}{$3\gets$Sheila(x)\\ $1\gets$believes(x,p)\\
$1\gets$p:\pdrs{2}{$4\gets$y~~$5\gets$z}{$4\gets$Chuck(y)\\ $5\gets$psychopath(z)\\
$5\gets$x=y\\ $2\gets$fit-to-watch-kids(y)}{$2<4$~~$2<5$~~$5=4$}}{$1< 3$}

}

\noindent\parbox[b]{\textwidth}{\textbf{NRRC}:
\ex. Ames\underline{, who stole from the FBI,} is behind bars.\\
\pdrs{1}{$3\gets$x~~$4\gets$y}{$3\gets$Ames(x)\\ $4\gets$FBI(y)\\ $5\gets$stole\_from(x,y)\\
$1\gets$behind\_bars(x)}{$1<3$~~$1<5$~~$5<4$~~$5=3$}

}

\noindent\parbox[b]{\textwidth}{\textbf{\textit{As}-parenthetical}:
\ex. It is not the case that Ames\underline{, as the press reported,} is a spy.\\
\textit{cf.} Ames, of whom the press reported he is a spy, is not (a spy).\\
\pdrs{1}{}{$1\gets\neg$\pdrs{2}{$3\gets$x~~$4\gets$y~~$5\gets$p}{$3\gets$Ames(x)\\ $4\gets$press(y)\\
$5\gets$p:\pdrs{6}{}{$6\gets$spy(x)}{$6<3$}\\ $5\gets$reported(y,p)\\ $2\gets$spy(x)}{$2<3$~~$2<5$~~$5<4$~~$5=3$}}{}

}
%[..]
%
%
%Note that \citet{potts2005logic} also introduces a \textsc{comma} operator, although
%he uses it to add supplemented content to the CI dimension.
%\citet{amaral2007review} argue against the use of a \textsc{comma} operator by
%adhering to both its overinterpretation (not all comma's are indicators of
%CIs -- not even all comma's that introduce a subordinate clause to a noun),
%and underinterpretation (since not all CIs are triggered by a comma
%\citep[cf. the `malheureusement' example by][]{bonami2004adverb}, and
%because of type-matching).
