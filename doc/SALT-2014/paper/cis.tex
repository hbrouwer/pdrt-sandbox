\documentclass{salt}

% This document template was developed by the editors of /Semantics and
%  Pragmatics/ and minimally modified to work with the salt.cls document
%  class.

%=====================================================================
%============================= packages ==============================

\usepackage{natbib}
\usepackage{linguex}
\renewcommand\firstrefdash{}

%=====================================================================
%============================== macros ===============================

\newcommand{\drs}[2]
{\footnotesize\begin{tabular}{c}
  \noalign{\smallskip}
  \begin{tabular}{|l|}
    \hline #1\\ \hline #2\\ \hline
  \end{tabular}
  \smallskip
\end{tabular}
\normalsize}

\newcommand{\pdrs}[4]
{\footnotesize\begin{tabular}{c}
  \noalign{\smallskip}
  $#1$\\
  \begin{tabular}{|l|}
    \hline #2\\ \hline #3\\ \hline #4 \\ \hline
  \end{tabular}
  \smallskip
\end{tabular}
\normalsize}

%=====================================================================
%========================= preamble material =========================

% Metadata for the PDF output. ASCII-only!
\pdfauthor{Noortje J. Venhuizen, Johan Bos, Petra Hendriks, Harm Brouwer}
\pdftitle{How and why conventional implicatures project}
\pdfkeywords{conventional implicatures, CIs, DRT, PDRT, projection}

% Optional short title inside square brackets, for the running headers.
% If no short title is given, no title appears in the headers.
\title[How and why CIs project]{% in sentence case (down style)
  How and why conventional implicatures project 
  \thanks{We thank \ldots}}

% Optional short author (last name only) inside square brackets, for the running headers.
% If no short author is given, no authors print in the headers.
\author[Venhuizen et al.]{% As many authors as you like, each separated by \AND.
  \saltauthor{Noortje J. Venhuizen \\ \institute{University of Groningen}} \AND
  \saltauthor{Johan Bos \\ \institute{University of Groningen}} \AND
  \saltauthor{Petra Hendriks \\ \institute{University of Groningen}} \AND
  \saltauthor{Harm Brouwer \\ \institute{University of Groningen}}%
}

%=====================================================================

\begin{document}

%=====================================================================
%============================ frontmatter ============================

\maketitle

%--------------------------------------------------------------------
% First page headers and page numbers
%
% the page number of the first page of this paper
% \setcounter{page}{1}

% Create the first page headings.
% This needs to be issued *after* \maketitle.
%       {volume}{first page}{last page}{year}{not used}{not used}
\firstpageheadings{24}{000}{000}{2014}{}{}
%
%
%---------------------------------------------------------------------

\begin{abstract}  
  Abstract here
\end{abstract}

\begin{keywords}
  conventional implicatures, CIs, DRT, PDRT, projection %(special formatting is fine)
\end{keywords}

\tableofcontents %%XXX comment out in final version

%=====================================================================
%============================ article text ===========================

\section{Introduction}\label{Introduction}

%XXX 

\begin{tabular}{p{0.2\textwidth} p{0.8\textwidth}}
  \textbf{Challenge:} & CIs provide \textit{new} information, but they
                        project as if providing \textit{old} information
                        (like presuppositions).\\

   \textbf{Solution:} & CIs provide \textit{new} information to the
                        (projecting) context created by a presupposition,
                        namely the one provided by its anchor.\\

\textbf{Implementation:} & CIs are projection-anaphoric in the sense that
                        they inherit their projection context from their
                        (projecting) anchor. This is formalized in PDRT
                        \citep{venhuizen2013iwcs} by introducing a dependency
                        at the lexical level between the projection site of
                        the CI and the projection site of the anchor.
\end{tabular}



\section{An analysis of supplemental CIs}

One of the most prominent accounts of presuppositions is the one by
\citet{sandt1992presupposition}, who treats presupposition projection as
anaphora resolution.
%%explain more.

CIs share their projection behaviour with other projection phenomena, so this
requires a unified analysis of their behaviour.

%\citet{potts2005logic} describes the following properties of CIs:
%
%\begin{itemize}
%  \item CIs tend to project globally (scopelessness)
%  \item CIs are non-cancellable
%  \item CIs are non-restrictive
%  \item Speaker-orientedness
% \end{itemize}

We will argue that all these properties can be described in terms of
a species of indexicality \citep[cf.][]{amaral2007review}, which in the case
of supplemental CIs comes down to indexicality to the projection site of the
anchor.

\subsection{Projection as anchoring}

\subsubsection{CIs are `piggybacking' on their anchor}

* CIs always occur with an anchor\\
* Similarity to E-type pronouns\\
* CIs always project as far as their anchor
%* CIs provide novel information to the context of their anchor


\subsubsection{CIs require their anchor to project}

Infelicitous with non-specific indefinites, quantificational constructions

The information structure of CIs (i.e., their subordinatedness) signals
projection; this is why we can distinguish between RRCs and NRRCs --- by
means of the comma in this case. Non-restrictiveness requires projection.
This is because a non-projecting anchor requires the introduction of a new
entity, which by definition lacks a complete description, so any new
information will contribute to the description of the referent and will
therefore be restrictive. On the other hand, in case the anchor projects,
this means that the referent has been established, in the sense that it can
be identified in the common ground, so providing new information does not
affect the choice of referent, so it works non-restrictively.

\ex. example with indefinite anchor -> only specific reading

\ex. example with pronominal anchor -> only felicitous if referent can be
  determined unambiguously.

Because of the requirement for a specific anchor, CIs are often attached to
presuppositional anchors, which explains their tendency to project to the
global discourse context. However, there are also examples of CIs that do
not project to the global discourse level, as in \Next \citep[example
from][]{amaral2007review}.

\ex. Joan is crazy. She's hallucinating that some geniuses in Silicon Valley
have invented a new brain chip that's been installed in her left temporal
lobe [\ldots]. Joan believes that her chip\underline{, which she has
installed last month,} has a twelve year guarantee.

In this example, the CI triggered by the underlined NRRC does not project to
the global discourse context, since it is clear that the speaker does not
want to convey that Joan actually had the chip installed last month. Note,
however, that the direct anchor of the CI content (``her chip''), is
a possessive construction --- a presupposition that in the current context
binds to the introduction of ``a new brain chip'' in the sentence before
\citep[following][]{sandt1992presupposition}.  Thus, the anchor is not
accommodated to the global discourse level, and therefore the CI attached to
the anchor isn't either. Yet, the fact that the anchor projects allows it to
felicitously occur with a (restrictive) CI.

%Something about generics? and inclusion and attribution as a contrast to
%identification with anchor? \cite[see][]{heringa2012appositional}

\subsection{Formalizing projection-anaphoricity}

\subsubsection{Projective Discourse Representation Theory}

[short introduction]

\subsubsection{Enhancing PDRSs with information structure}

* Adding MAPs to PDRSs. \\
* Projection as \textit{strict} subordination


\subsection{Representing the projection behaviour of CIs}

\subsubsection{Supplemental clauses}

\paragraph{Appositives.}~

\noindent \parbox[b]{\textwidth}{
\ex. Sheila believes that Chuck\underline{, a psychopath,} is fit to watch
the kids.\\
\pdrs{1}{$3\gets$x~~$1\gets$p}{$3\gets$Sheila(x)\\ $1\gets$believes(x,p)\\
$1\gets$p:\pdrs{2}{$4\gets$y~~$5\gets$z}{$4\gets$Chuck(y)\\ 
$5\gets$psychopath(z)\\ $5\gets$z=y\\ $2\gets$fit-to-watch-kids(y)}{$2<4$
~~$2<5$~~$5=4$}}{$1< 3$}

}

\paragraph{NRRCs.}~

\noindent\parbox[b]{\textwidth}{
\ex. Ames\underline{, who stole from the FBI,} is behind bars.\\
\pdrs{1}{$3\gets$x~~$4\gets$y}{$3\gets$Ames(x)\\ $4\gets$FBI(y)\\ 
$5\gets$stole\_from(x,y)\\ $1\gets$behind\_bars(x)}{$1<3$~~$1<5$~~$5<4$
~~$5=3$}

}

\paragraph{\textit{As}-parentheticals.}~

\noindent\parbox[b]{\textwidth}{
\ex. It is not the case that Ames\underline{, as the press reported,} is
a spy.\\
\textit{cf.} Ames, of whom the press reported he is a spy, is not (a spy).\\
\pdrs{1}{}{$1\gets\neg$\pdrs{2}{$3\gets$x~~$4\gets$y~~$5\gets$p}{
$3\gets$Ames(x)\\ $4\gets$press(y)\\ $5\gets$p:\pdrs{6}{}{$6\gets$spy(x)}{
$6<3$}\\ $5\gets$reported(y,p)\\ $2\gets$spy(x)}{$2<3$~~$2<5$~~$5<4$
~~$5=3$}}{}

}
%[..]
%
%
%Note that \citet{potts2005logic} also introduces a \textsc{comma} operator,
%although he uses it to add supplemented content to the CI dimension.
%\citet{amaral2007review} argue against the use of a \textsc{comma} operator
%by adhering to both its overinterpretation (not all comma's are indicators
%of CIs -- not even all comma's that introduce a subordinate clause to
%a noun), and underinterpretation (since not all CIs are triggered by
%a comma \citep[cf. the `malheureusement' example by][]{bonami2004adverb},
%and because of type-matching).
%
%
\subsubsection{Supplemental adverbs}

\paragraph{Topic/Speaker-oriented Adverbs.}


\paragraph{Utterance level modifiers.} (e.g., ``Frankly, I don't like him'',
``In case you're hungry, I'm going to the grocery.'', etc.) Following
\citet[pp.725-729]{amaral2007review}, these modifiers behave differently
than other CIs, both on the theoretical and the implementational level.


%


\section{Beyond supplements}

\subsection{Speaker-orientedness in PDRT}

Following the traditional model-theoretic interpretation of DRT \citep[see,
e.g.][]{kamp1981theory,kamp1993discourse,kamp2011discourse}, a Discourse
Representation Structure (DRS) is thought of as a \textit{partial model}
representing the information conveyed by some utterance, which needs to be
embedded in a \textit{total model} to determine its interpretation. This
embedding is done via an \textit{embedding function} from the set of
discourse referents in the universe of the DRS $K$ into elements in the
domain of the total model $M$, in such a way that all conditions of $K$ are
verified in $M$ (with respect to some world $w$).

To implement speaker-orientation, we can assume that each speaker $S$ has
its own model $M_S$ describing the current state of affairs of the world.
These models may or may not coincide with the model representing the
\textit{actual} state of affairs, which is represented by the `Truth Model'
$M_T$. The idea is that the interpretation a PDRS condition may be
restricted to some (subjective) model, meaning that it only needs to hold in
the designated model. Since in PDRT the information about the interpretation
of linguistic content is explicitly available (in the form of
\textit{pointers}), an interpretational restriction with respect to some
model may also be incorporated as part of the pointer, e.g. via a subscript.
This thus means that for some condition with a pointer of the form `$p_S$',
it holds that this condition is only required to hold with respect to the
model of speaker $S$. Thus, when determining the truth of some utterance
containing some subjective content (i.e., with respect to the Truth Model
$M_T$), the subjective content has to be fulfilled only with respect to the
subjective model $M_S$; in case this holds, the entire utterance will come
out true (provided, of course, that all other conditions are verified with
respect to $M_T$). Note that this places a constraint on the domains of
$M_S$ and $M_T$ insofar that they must overlap with respect to the
referent(s) mentioned in the subjective condition.

\subsection{Expressives}

We focus on supplemental CIs since these are the most common and
well-studied type of CI.  However, a case can be made for a similar account
of expressives \citep[including expressive attributive adjectives, epithets,
honorifics, and tense-variations such as the German `Konjunktiv I';
cf.][]{potts2005logic}. The status of expressives has been open to some
debate \citep[...]{potts2004japanese,geurts2007fucking}.  But like
supplemental CIs, expressives in general require some antecedent (or:
anchor) to express some opinion about.
%%examples (see, e.g. Amaral et al. p. 711)
However,
the notion of speaker-orientedness is especially important in this context,
so a framework is required that can incorporate speaker attitudes.
[...]
%Q: Do expressives require a projecting anchor?

\paragraph{Epithets.} An epithet is a referential noun phrase that conveys
an additional subjective mood of the speaker toward the object of reference.

%\citep[cf.][]{kratzer1999beyond}
\noindent\parbox[b]{\textwidth}{
\ex. Father$_F$ said he would not allow me$_S$ to hit/marry \underline{that bastard} Webster.\\
  \pdrs{1}{$5\gets$x~~$1\gets$p}{$5\gets$Father(x)\\ $1\gets$said(x,p)\\
    $1\gets$p:\pdrs{2}{}{$2\gets\neg$\pdrs{3}{$3\gets$e~~$6\gets$y}{
      $3\gets$allow(x,e)\\ $3\gets$hit/marry(e)\\ $3\gets$Agent(e,S)\\ 
      $3\gets$Patient(e,y)\\ $6\gets$Webster(y)\\ $7_{S/F}\gets$bastard(y)}{
        $3<6$~~$3<7$~~$7=6$}}{}}{$1<5$}

}


\noindent\parbox[b]{\textwidth}{
\ex. Every democrat advocating a proposal for reform says 
  \underline{the stupid thing} is worthwhile.\\
%a.~~\pdrs{1}{}{$1\gets$\pdrs{2}{$2\gets$x~~$2\gets$y}{$2\gets$Democrat(x)\\ 
%  $2\gets$proposal\_for\_reform(y)\\ $2\gets$advocate(x,y)}{}$\Rightarrow
%  $\pdrs{3}{$3\gets$p}{$3\gets$says(x,p)\\ 
%  $3\gets$p:\pdrs{4}{$4,5_S\gets$z}{$5_S\gets$stupid\_thing(z)\\ $4,5_S\gets$z=y\\
%  $4\gets$worthwhile(z)}{$4<5$~~$5=2$}}{$3<2$}}{}
  \pdrs{1}{}{$1\gets$\pdrs{2}{$2\gets$x~~$5\gets$y}{$2\gets$Democrat(x)\\ 
    $5\gets$proposal\_for\_reform(y)\\ $2\gets$advocate(x,y)}{
      $2<5$}$\Rightarrow$\pdrs{3}{$3\gets$p}{$3\gets$says(x,p)\\ 
        $3\gets$p:\pdrs{4}{}{$6_S\gets$stupid\_thing(y)\\ 
        $4\gets$worthwhile(y)}{$4<6$~~$6=5$}}{$3<2$}}{}

}

\noindent 
%The epithet ``the stupid thing'' is interpreted in \Last[a] as
%contributing both an asserted, anaphoric referent for the adjective
%``worthwhile'', and a subjective condition on this referent. By making use
%of multiple pointers (which in terms of interpretation comes down to
%duplicating the declaration of the referent and the condition), the  dual
%role of the expressive noun phrase becomes explicit: it both provides an
%antecedent for the asserted statement, and conveys the meaning of the
%speaker about this antecedent. Note that the antecedent (``a proposal for
%reform'') is itself not a projecting referent, but since the epithet serves
%as an anaphor, the object of expression is projected, just as in the case of
%supplemental CIs. 
%%explain better
The representation in \Last (where the referent equality is eliminated)
represents the specific reading of the indefinite noun phrase ``a proposal
for reform''. It may be argued that the use of an expressive requires
(presupposes) a projecting anchor, since it only makes sense to express
feelings about some uniquely identifiable object. 

\paragraph{Honorifics.} Honorifics (as in Japanese) are morphemes that
express the social status of the addressee relative to the speaker.

\noindent\parbox[b]{\textwidth}{
\ex. Ame~~ga~~~furi-\underline{mashi}-ta.\\
\textit{rain} \textsc{subj} \textit{fall}-\textsc{hon}-past\\
`It rained.' (\textit{performative honorific})\\
\pdrs{1}{$2\gets$e}{$2\gets$rain(e)\\ $1\gets$fell(e)\\
  $3_S\gets$\textsc{hon}(e)}{$1<2$~~$1<3$~~$3=2$}

}

\noindent The requirement for a projecting anchor is even clearer in the
case of honorifics, since they always serve as definite descriptions that
refer to a specific referent \citep[cf.][]{potts2004japanese}. Therefore,
the `raining event' in \Last is interpreted as a presupposition (referring
to the general concept of rain), so that the performative honorific can be
correctly interpreted by means of projection (again, the referential
equality is resolved).

\paragraph{German Konjunktiv I.} The German language has a special
subjunctive mood that expresses lack of speaker commitment. Thus, in the
example below, the speaker of the sentence is not convinced that Sheila is
actually sick.

\noindent\parbox[b]{\textwidth}{
\ex. Sheila behauptet dass sie krank \underline{sei}.\\
\textit{Sheila maintains that she sick~~~be}-\textsc{konj}\\
`Sheila maintains/claims that she is sick.'\\
\pdrs{1}{$4\gets$x~~$1\gets$p}{$4\gets$Sheila(x)\\ $1\gets$behauptet(x,p)\\
  $1\gets$p:\pdrs{2}{}{$2\gets$krank(x)}{$2<4$}\\ 
  $5_S\gets\neg$\pdrs{3}{}{$3\gets$krank(x)}{$3<4$}}{$1<4$~~$1<5$~~$5=4$}

}

\noindent In order to represent both the asserted content and the expressive
contribution of this example, the content modified by the subjunctive mood
needs to be duplicated (similar to the case of the
\textit{As}-parentheticals). 

\paragraph{Expressive attributive adjectives.} The class of expressive
attributive adjectives conveys the speaker's attitude toward someone of
something via the use of certain adjectives.

\noindent\parbox[b]{\textwidth}{
\ex. Sue's dog is \underline{fucking} mean.\\
a.~~~\pdrs{1}{$2\gets$x~~$3\gets$y}{$2\gets$Sue(x)\\ $3\gets$dog(y)\\
  $3\gets$of(y,x)\\ $1\gets$mean(y)\\ $4:S\gets$fucking\_mean(y)}{
  $1<2$~~$1<3$~~$1<4$~~$4=3$}
~~~b.~~~\pdrs{1}{$2\gets$x~~$3\gets$y~~$1\gets$e}{$2\gets$Sue(x)\\ $3\gets$dog(y)\\
  $3\gets$of(y,x)\\ $1\gets$mean(e)\\ $1\gets$Topic(e,y)\\
  $1_S\gets$fucking(e)}{$1<2$~~$1<3$}

}

\noindent In PDRS \Last[a], the anchor of the CI is ``Sue's dog'', which
projects.  However, the representation is not completely satisfactory, since
the adjective ``mean'' is duplicated: it occurs once as part of the asserted
content, and once as part of the expressive CI content. Moreover, the
contraction of ``fucking'' and ``mean'' serving as one predicate goes
against the compositional idea in (P)DRT. In \Last[b], this tension is
resolved by using a neo-Davidsonian event-structure for the adjectival
construction; by letting the adjective ``mean'' introduce an event, which is
related to the subject using VerbNet roles \citep{kipper2008large}, the
expressive adjective ``fucking'' can manipulate the event itself, instead of
the subject. This provides a more intuitive distinction between the asserted
content and the expressive content of this utterance. However, the direct
correspondence to supplemental CIs, which require a projecting anchor, seems
to be obfuscated with this representation.



\section{Discussion}

\subsection{Evaluating the analysis}

\subsection{Projection and at-issueness}

\citet{simons2010projects}

\subsection{Non-supplemental CIs: Expressives}

We focus on supplemental CIs since these are the most common and
well-studied type of CI.  However, a case can be made for a similar account
of expressives \citep[including expressive attributive adjectives, epithets,
honorifics, and tense-variations such as the German `Konjunktiv I';
cf.][]{potts2005logic}. The status of expressives has been open to some
debate \citep[...]{potts2004japanese,geurts2007fucking}.  But like
supplemental CIs, expressives in general require some antecedent (or:
anchor) to express some opinion about; this antecedent may have the form of
a noun phrase, an adjective, a proposition, or even a verb.
%%examples (see, e.g. Amaral et al. p. 711)
Each of these antecedents may result in different projection patterns, which
may be explained by the semantics of the different antecedents. However,
the notion of speaker-orientedness is especially important in this context,
so a framework is required that can incorporate speaker attitudes.

\subsection{Speaker-orientedness}



\input{conclusion}

%\section{Section}
%\subsection{Subsection}
%\subsubsection{Subsubsection}
%\paragraph{Paragraph}
%\subparagraph{Subparagraph}
%
%\begin{table}
%  \begin{tabular}{c|c|c}
%     1 & 2 & 3 \\
%     la & di & da \\
%  \end{tabular}
%  \caption{An example of table}
%  \label{my_table}
%\end{table}

%=====================================================================

\newpage %%XXX comment out in final version
\bibliography{../../presupposition}

%=====================================================================

\begin{addresses}
  \begin{address}
    Noortje J. Venhuizen \\
    CLCG, University of Groningen\\
    Oude Kijk in 't Jatstraat 26\\
    9712 EK Groningen\\
    The Netherlands \\
    \email{n.j.venhuizen@rug.nl}
  \end{address}
  \begin{address}
    Johan Bos \\
    CLCG, University of Groningen\\
    Oude Kijk in 't Jatstraat 26\\
    9712 EK Groningen\\
    The Netherlands \\
    \email{johan.bos@rug.nl}
  \end{address}
  \begin{address}
    Petra Hendriks \\
    CLCG, University of Groningen\\
    Oude Kijk in 't Jatstraat 26\\
    9712 EK Groningen\\
    The Netherlands \\
    \email{p.hendriks@rug.nl}
  \end{address}
  \begin{address}
    Harm Brouwer \\
    CLCG, University of Groningen\\
    Oude Kijk in 't Jatstraat 26\\
    9712 EK Groningen\\
    The Netherlands \\
    \email{harm.brouwer@rug.nl}
  \end{address}
\end{addresses}

%=====================================================================

\clearpage

%XXX 
\bigskip
\noindent
\begin{tabular}{p{0.2\textwidth} p{0.75\textwidth}}
          \textbf{Goal:} & A unified account of projection phenomena.\\
   \textbf{Challenge I:} & CIs provide \textit{new} information, but they
                           project as if providing \textit{old} information
                           (like presuppositions).\\
   \textbf{Observation:} & Supplemental CIs occur with a projecting anchor,
                           and always project as far as their anchor.\\
    \textbf{Solution I:} & CIs provide \textit{new} information to the
                           (projecting) context created by a presupposition,
                           namely the one provided by its anchor.\\
\textbf{Implementation:} & CIs are projection-anaphoric in the sense that
                           they inherit their projection context from their
                           (projecting) anchor. This is formalized in PDRT
                           \citep{venhuizen2013iwcs} by introducing a
                           dependency at the lexical level between the
                           projection site of the CI and the projection site
                           of the anchor.\\
     \textbf{Next Step:} & Other CIs: expressives and supplemental adverbs.\\
  \textbf{Challenge II:} & Incorporating subject-orientedness in (P)DRT.\\
   \textbf{Solution II:} & Subjective PDRT: using pointers to restrict speaker
                           models.\\
    \textbf{Discussion:} & Are CIs a homogeneous class? What is their relation
                           to presuppositions and anaphora?\\
\end{tabular}\\

References: \citep{delgobbo2003appositives,potts2005logic,amaral2007review,
  nouwen2007appositives,harris2009perspective,heringa2012appositional,
  schlenker2013supplements,nouwen2014note,potts2013presupposition}

\end{document}
