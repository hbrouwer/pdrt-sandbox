\section{Translation and interpretation}\label{sec:translating}

Now that we have defined how DRSs and PDRSs are constructed and combined, we
turn to their interpretation. Just as in standard descriptions of DRT, the
interpretation of PDRSs can be given directly in terms of a model-theoretic
interpretation, or indirectly, via a translation to first-order logic, which
for PDRSs also involves a translation to DRSs. An algorithmic PDRS to DRS
translation can be found in \cite{venhuizen2013iwcs}. Here, we formulate
a translation via an intermediate representation between PDRSs and DRSs: the
Projection Table. In this section, we define Projection Tables (in short,
PTables), describe a translation from PDRSs to PTables, and from PTables to
DRS. Accordingly, we present the outlines of a direct model-theoretic
interpretation of DRSs and PDRSs. 
%%XXX echt??

\subsection{Projection Table}

A Projection Table (PTable) contains the same information as a PDRS, but has the
advantage of explicitly separating information about content and form, and
providing a `flat' (i.e., non-recursive) representation. 
%% some more explanation?
Besides providing
a suitable intermediate representation for operations on PDRSs, such as
a translation to DRSs, these flat representations can be shown to provide
a good candidate for computational tasks such as surface realisation
\citeaffixed{basile2013aligning}{see, e.g.}.

A PTable is an $n\times3$ matrix, which can be represented as a set of
triples $[C,I,P]$, where $C$ is some PDRS content (a referent, a predicate
or an operator with one or more PDRS labels), $I$ is the introduction site
of the content, and  $P$ is the projection site of the content. The
translation from a PDRS to a projection table can be straightforwardly
defined using two operations $\mathcal{T}$ and $\mathcal{I}$, defined in the
following seven clauses (here, $\Omega_1$ indicates a unary operator: i.e.,
one of $\{\neg, \Box, \Diamond$\}; and $\Omega_2$ a binary operator; one of
$\{\Rightarrow, \vee\}$).

\begin{definition}[PDRS to Projection Table]~\\\vspace{-12pt}
\begin{enumerate}
  \item $\mathcal{T}(\langle l,\{\mu_1...\mu_n\},
    \{\delta_1...\delta_m\}, \{\chi_1...\chi_k\}\rangle$)\\
    = $\mathcal{I}(\mu_1,l)\cup...\cup \mathcal{I}(\mu_n, l)\cup
      \mathcal{I}(\delta_1,l)\cup...\cup \mathcal{I}(\delta_m,l)\cup
      \mathcal{I}(\chi_1,l)\cup...\cup \mathcal{I}(\chi_k,l)$
      %What about the MAPs?
  \item $\mathcal{I}(\langle v_1, v_2 \rangle,l)
    = \{[\langle v_1, v_2 \rangle,l,l]\}$
  \item $\mathcal{I}(\langle v_i, x_i \rangle,l)
    = \{[x_i,l,v_i]\}$
  \item $\mathcal{I}(\langle v_i, R(x_1...x_j) \rangle,l)
    = \{[R(x_1...x_j),l,v_i]\}$
  \item $\mathcal{I}(\langle v_i, \Omega_1 K_j) \rangle,l)
    = [\Omega_1 (lab(K_j)),l,v_i] : \mathcal{T}(K_j)$
  \item $\mathcal{I}(\langle v_i,  \Omega_2(K_j, K_k) \rangle,l)
    = [\Omega_2(lab(K_j),lab(K_k)),l,v_i] : \mathcal{T}(K_j)
      \cup \mathcal{T}(K_k)$
  \item $\mathcal{I}(\langle v_i, x_i:K_j \rangle,l)
    = [x_i:lab(K_j),l,v_i] : \mathcal{T}(K_j)$
\end{enumerate}
\end{definition}

\noindent Let's show an example:

\ex. It is not the case that John is a vegetarian, he's a vegan.

\hspace*{0.8cm}\begin{minipage}{\linewidth}
\begin{multicols}{2}
  \a. \hspace*{-0.3cm}{
        \pdrs{1}{}{
          $1\gets\neg$\pdrs{2}{$3\gets x$}{
            $3\gets$John($x$)\\ $2\gets$vegetarian($x$)
            }{$2 \leq 3$}\\
          $1\gets$vegan($x$)
        }{$1\leq 3$}}
        \b. \vspace*{1pt}\small \begin{tabular}{| l | c | c |}
      \hline
      Content & I & P\\
      \hline
     $\neg$ 2 & 1 & 1\\
            x & 2 & 3\\
      John(x) & 2 & 3\\
        (2,3) & 2 & 2\\
vegetarian(x) & 2 & 2\\
     vegan(x) & 1 & 1\\
        (1,3) & 1 & 1\\
      \hline
    \end{tabular}\\\vspace*{1pt}

\end{multicols}
\end{minipage}\\

\noindent %%

To reconstruct a PDRS from a projection table, we need to
retrieve the information about the structure of the original PDRS,
and then insert all other PDRS content at the correct introduction site. For
the first step, we need to know the label of the outer PDRS, and the labels
of all sub-PDRSs. The latter information can be simply derived from the PDRS
content represented in \textsc{ptable} $T$: \textsc{sub-pdrss} $=\{l~|~\Omega_1
  l \in \bigcup C ~\vee~ \Omega_2(l,l') \in \bigcup C ~\vee~ \Omega_2(l',l) \in
\bigcup C ~\vee~ x:l \in \bigcup C \}$, where $\bigcup C$ represents all
PDRS content represented in \textsc{ptable} $T$. Since the label of the
outer PDRS by definition is the only label that occurs as an introduction
site but not as a sub-PDRS, it can be retrieved as follows:
\textsc{outer-label} $=\{i~|~i\in\bigcup I \wedge i \notin
\textsc{sub-pdrss}\}$, where $\bigcup I$ is the collection of all
introduction sites in $T$.

With this information, we can re-create the original PDRS starting from an
empty PDRS with the \textsc{outer-label} as label. Then all other content
can be incrementally inserted on the basis of the introduction site
associated with the PDRS content. When translating a PDRS to a DRS, all
content needs to be moved to the projection site, instead of the
introduction site, since the information provided by the pointers will be
lost, so all content needs to occur at its interpretation site in order to
obtain the desired truth-conditional interpretation. One constraint on this
operation is the occurrence of free pointers the projection site of
presupposed content. These free contexts can be retrieved from the
projection table by collecting all projection contexts that do not occur
within the \textsc{sub-pdrss} or as the \textsc{outer-label} of $T$:
\textsc{free-contexts} $=\{p~|~p\in\bigcup P \wedge p \notin
\textsc{sub-pdrss} \wedge p \notin \textsc{outer-label}\}$, where $\bigcup
P$ is the collection of all projection sites in $T$.
%what about an empty PDRS? With this def of projection table it is
%impossible to retrieve its label.. Idea: Maybe add recursive accessibility
%of label in the same way as maps? Yes!

In order to determine the interpretation site of such presupposed content,
the \textsc{free-contexts} need to be translated into the labels from the
\textsc{sub-pdrss} or the \textsc{outer-label}. For this, we can use the
MAPs, which are available in the projection table as empty PDRS content.

\subsection{PDRS to DRS and FOL}

Step 1: convert PDRS $P$ into a proper, resolved, (simple) and plain PDRS $P'$
(directly, or via projection table)\\

\noindent Step 2: remove all projection variables from $P'$

%isFOLDRS :: DRS -> Bool
%Returns whether DRS d can be translated into a FOLForm.
%% isResolved & isPure & isProper

\begin{definition}[DRS to FOL]~\citeaffixed{bos2003implementing}{cf.}
  \begin{enumerate}
    \item $fo(\langle\{x_1,\ldots,x_n\}\{c_1,\ldots,c_m\}\rangle,w) 
      = \exists x_1\ldots\exists x_n(fo(c_1,w)\wedge\ldots\wedge fo(c_m,w))$
    \item $fo(R(x_i,\ldots,x_j),w) = R(w,x_i,\ldots,x_j)$
    \item $fo(\neg K,w) = \neg fo(K,w)$
    \item $fo(K_1 \vee K_2,w) = fo(K_1,w) \vee fo(K_2,w)$
    \item $fo(\langle\{x_1,\ldots,x_n\}\{c_1,\ldots,c_m\}\rangle \Rightarrow K,w)
      = \forall x_1 \ldots \forall x_n((fo(c_1,w) \wedge \ldots \wedge fo(c_m,w)) 
          \rightarrow fo(K,w))$
    \item $fo(\Diamond K,w) = \exists v (R(w,v) \wedge fo(K,v))$
    \item $fo(\Box K,w) = \forall v (R(w,v) \rightarrow fo(K,v))$
    \item $fo(v:K,w) = R(w,v) \wedge fo(K,v)$
  \end{enumerate}
\end{definition}


\subsection{Model-theoretic interpretation}

The interpretation of a (modal) DRS is defined relative to a model $M=\langle F,
I\rangle$, consisting of a frame $F=\langle W,R,U\rangle$, where $W$ is
a non-empty set of possible worlds, $R$ is an accessibility relation on $W$,
and $U$ is a set of individuals, and an interpretation function $I$ which
maps $n$-ary relations to $n$-ary tuples of elements from $U$.  The
interpretation of DRS $K$ is given in terms of an \textit{embedding
function} $f$, which verifies $K$ in $M$ relative to some world $w\in W$.
Intuitively, a DRS is considered to be a \textit{partial} model representing
the information conveyed by some piece of discourse; determining the truth
of DRS $K$ in some world $w$, therefore means finding a function that embeds
$K$ in the \textit{total} model $M$ with respect to $w$.  An embedding
function for a DRS $K$ in a model $M$ in world $w$, then, is a partial
function from the discourse referents in the universe $U_K$ of $K$ into
elements in the domain $U$ of $M$, such that all conditions of $K$ are
verified in $M$ with respect to $w$. 

The definition below defines when an embedding function $g$ verifies DRS $K$
with respect to $f$ given model $M$ and world $w$ (formally defined as: $\langle f,g\rangle
\vDash_{M,w} K$), and when $g$ verifies DRS condition $c$ given model $M$
and world $w$ (defined as: $g \vDash_{M,w} c$).  Here, the notation $f\subseteq_X g$
indicates that embedding function $g$ extends embedding function $f$ to the
discourse referents in the (possibly empty) set $X$ (i.e., $Dom(g) = Dom(f)
\cup X)$, such that for all referents $x$ in the domain $Dom(f)$ of $f$,
$f(x)=g(x)$. 

\begin{definition}[DRT Verifying Embeddings]\label{def:drtve}
  \citeaffixed{kamp2011discourse}{cf.}
  \begin{enumerate}
    \item $\langle f,g\rangle \vDash_{M,w} \langle U,C\rangle$ iff 
      $f\subseteq_U g$ and for all $c\in C$ it holds that $g \vDash_{M,w} c$
    \item $g \vDash_{M,w} P(x_1,\ldots,x_n)$ iff 
      $\langle g(x_1),\ldots,g(x_n)\rangle \in I(P)$
    \item $g \vDash_{M,w} \neg K$ iff there exists no $h$ such that 
      $\langle g,h \rangle \vDash_{M,w} K$
    \item $g \vDash_{M,w} K_1 \vee K_2$ iff there exists some $h$ such that 
      $\langle g,h \rangle \vDash_{M,w} K_1$ or $\langle g,h \rangle \vDash_{M,w} K_2$
    \item $g \vDash_{M,w} K_1 \Rightarrow K_2$ iff for all $h$ such that 
      $\langle g,h \rangle \vDash_{M,w} K_1$, there exists a $k$ such that 
      $\langle h,k \rangle \vDash_{M,w} K_2$
    \item $g \vDash_{M,w} \Box K$ iff for all $v\in W$ such that $R(w,v)$ 
      it holds that $g \vDash_{M,v} K$
    \item $g \vDash_{M,w} \Diamond K$ iff there exists a $v \in W$ such that
      $R(w,v)$ and $g \vDash_{M,v} K$
   \item $g \vDash_{M,w} v:K$ iff $R(w,v)$ and $g \vDash_{M,v} K$
  \end{enumerate}
\end{definition}

To give an example, say we have the DRS $K$ shown in \Next, and we want to
determine, given some model $M$ and world $w$, whether this DRS is true.

\ex. John is does not own a donkey.\\
      $K:$ \drs{$x$}{
        John($x$)\\
        $\neg$\drs{$y$}{donkey($y$)\\ owns($x,y$)}}

To determine the truth of $K$, we need to determine whether there exists an
embedding function $f$ that verifies $K$ given $M$ and $w$ with respect to
the empty assignment $\Lambda$, that is, we need to find an $f$ such that
$\langle \Lambda, f\rangle \vDash_{M,w} K$. Following the definition above,
this holds just in case $f$ extends $\Lambda$ to the universe of $K$ and $f$
verifies all conditions in $K$. Since $\Lambda$ is the empty assignment
function, this just means that the domain of $f$ is the universe of $K$,
i.e., $Dom(f)=\{x\}$, and the range of $f$ are those elements in the domain
$U$ of $M$ such that the conditions of $K$ come out true in $M$ given $w$.
The first condition in $K$ is verified by $f$ given $M$ and $w$ just in case
$f(x) \in I(John)$, and the second condition is verified in case there does
\textit{not} exist any embedding function $g$, such that $Dom(g)=\{x,y\}$
and $f(x)=g(x)$, and for which it holds that $g(y)\in I(donkey)$ and
$\langle g(x), g(y)\rangle\in I(owns)$.

In order to provide a model-theoretic interpretation for PDRSs, we need
a way to determine the domain and range of the verifying embeddings that
takes into account the projection of content from its introduction site to
its interpretation site. We can do this by defining the embedding function
over PDRS-contexts instead of PDRSs; as before, a PDRS-context may refer to a PDRS
itself, or a projected context created by the occurrence of a free pointer.
We define PDRS-contexts like DRSs; as tuples consisting of a set of referents and a set of conditions. 


%Here we show how these tuples can be derived from a (gobal) PDRS and a projection variable.
%%%XXX

%% see definition in PDRT-sandbox

%\begin{definition}[PDRS-context]  A PDRS-context in a PDRS $P$ is a set of
%  referents and a set of conditions derived from $P$ given a projection
%  variable $l$.
%  \begin{enumerate}
%    \item $\mathcal{PC}(l,P) = \langle U_{l,P},C_{l,P}\rangle$ where: $U_{l,P} = \{r ~|~ \langle l,r\rangle \in \mathcal{U}(P)\}$ and  $C_{l,P} = \bigcup_{c\in C} Con(c,l,P)$
%      %% NB U(P) is only the universe of P, and \mathcal{U}(P) is the union of all universes in P.
%    \item $Con(\langle p,R(x_1,...x_n) \rangle,l,P) 
%      = \{R(x_1,...x_n)\}$ if $p=l$; otherwise $=\{\}$
%    \item $Con(\langle p,\neg P_1\rangle,l,P) = \{\neg \mathcal{PC}(lab(P_1),P)\}$ if $p=l$; otherwise $=\{\}$
%    \item \ldots
%  \end{enumerate}
%\end{definition}

%%%XXX

We can determine how to traverse the PDRS-contexts
of a given PDRS based on the MAPs; this strategy assumes that the PDRS is
\textit{connected}, since all PDRS-contexts should be present in the graph
resulting from collecting all MAPs, and \textit{pure}, since otiose uses of
projection variables may result in ambiguities. Moreover, just like in the
case of DRT, we need to assume properness, since a (P)DRS cannot be verified
if it contains free occurrences of referents.

%%%XXX

\begin{enumerate}
  \item Collect all projected contexts (i.e, collect free projection
    variables using $\mathcal{F_V}$; see Definition~\ref{def:fpvs}) and
    their order (using MAPs);
  \item Define embedding functions for these projected PDRS-contexts 
    (taking into account the order provided by the maps) in line with
    DRT's verifying embeddings, cf. Definition~\ref{def:drtve};
  \item Define embedding functions that extend the ones for the projection
    contexts for the PDRS-context of the global PDRS and its (asserted)
    subPDRSs, in line with DRT models.
\end{enumerate}



%\begin{definition}[PDRT Verifying Embeddings]~
%  \begin{enumerate}
%    \item $\langle f,g\rangle \vDash_{M,w} P$ iff for all $l$ such that $l
%      \in\{l\} \cup \bigcup\textsc{maps}(P)$ it holds that there exist a $h_1$ and
%      $h_2$, such that $f \subseteq h_1$ and $h_2 \subseteq g$ and $\langle
%      h_1,h_2\rangle \vDash_{M,w} \mathcal{PC}(l,P)$, and for all $m$ such that
%      $(l,m)\in\textsc{maps}(P)$ and $\langle h_3,h_4\rangle \vDash_{M,w}
%      \mathcal{PC}(m,P)$ it holds that $h_4 \subseteq h_1$.
%    \item $\langle f,g\rangle \vDash_{M,w} \langle U_{l,P},C_{l,P}\rangle$ iff $f \subseteq_{U_{l,P}} g$ and for all $c\in C_{l,P}$ it holds that $g\vDash_{M,w} c$
%    \item[3-9.] see Definition~\ref{def:drtve}:2-8.
%  \end{enumerate}
%\end{definition}


