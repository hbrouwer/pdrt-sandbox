\section{Projective DRT}\label{sec:pdrt}

The goal of creating a semantic representation of some linguistic utterance
is to capture the meaning that is conveyed by that utterance.  This may
include only the literal meaning, or the enriched as resulting from
pragmatic reasoning \citep[cf.][]{grice1975logic}. In this paper, we will
focus on the literal meaning, of which projected content is argued to be
a crucial part.  As a new formalism, we propose a new extension of DRT,
called Projective DRT \citep[PDRT;][]{venhuizen2013pdrt}.  In PDRT, all
referents and conditions are associated with a variable (a \emph{pointer})
that indicates their accommodation site by means of binding to \emph{context
labels} associated with each PDRS. This way, the only difference between
projected and asserted content is the context in which the pointer of some
content is bound; for asserted content this will be the local context, while
projected material may point to a higher context.

In this section, we will introduce the formal framework of PDRT and show how
it can be used to create a parsimonious representation of projected and
non-projected content.  We define a compositional construction of PDRT's
representation structures, and show how the resulting representations are
computationally efficient and user-friendly, because they stay close to the
linguistic surface structure.  The main difference between asserted,
presupposed and conventionally implied content in PDRT is the way in which
the content is integrated (resolved) into the representation of the
foregoing context.  We formulate three basic resolution strategies:
\emph{local accommodation}, \emph{global accommodation} and \emph{binding}.
In section~\ref{sec:categorization}, we show that the treatment of
presuppositions and conventional implicatures in PDRT inspires a general
classification of projected and asserted content in terms of their
compositional behaviour.

\subsection{Projection as variable binding}

Following traditional Discourse Representation Theory
\citep[DRT,][]{kamp1993discourse}, the semantic representation of
a discourse is constituted by recursive structures, that contain a set of
discourse referents and a set of conditions on these referents. These
structures can be seen as embedded \emph{contexts}, which represent the
logical form of the discourse.  In Projective DRT, the representation
structures are called PDRSs, and each PDRS introduces a unique \emph{label}.
All referents and conditions of a PDRS are associated with a \emph{pointer},
which is used to indicate in which context the material is interpreted by
means of binding the pointer to a context label.  In the default case, the
pointer of some content is bound by the label of its local PDRS, so content
is interpreted locally, thus indicating asserted material. In the case of
projected content, however, the content should be interpreted in some higher
context; this is indicated by either binding the pointer to a higher,
accessible PDRS label, or by means of a free pointer.  Pointing to a higher
PDRS means that the content is interpreted within the context of that PDRS.
A free variable as pointer indicates that no interpretation site has been
determined yet, so the content is interpreted at the global discourse level.
In case more context becomes available, the free pointers of a given
representation may still become bound; this nicely captures the behaviour of
accommodated presuppositions, which are interpreted at the global discourse
context, but may still be bound when more context becomes available.

Example \Next[a] shows the PDRS of a sentence
with only asserted content and \Next[b] shows the PDRS of a sentence with
a presupposition trigger. The labels introduced by the PDRSs are shown on
top of each PDRS.  The pointers associated with referents and conditions are
indicated using the $\gets$-operator. 

\begin{flushleft}
\begin{minipage}{0.85\linewidth}
\begin{multicols}{2}
\ex. \a. Nobody sees a man.\\
\hspace*{-0.3cm}{\small
  \pdrs{1}{}{
    $1\gets\neg$\pdrs{2}{$2\gets x, 2\gets y$}{
      $2\gets$person($x$)\\ $2\gets$man($y$)\\ $2\gets$see($x,y$)
    }
  }
}
\b. Nobody sees John.\\
\hspace*{-0.3cm}{\small
  \pdrs{1}{}{
    $1\gets\neg$\pdrs{2}{$2\gets x, f\gets y$}{
      $2\gets$person($x$)\\ $f\gets$John($y$)\\ $2\gets$see($x,y$)
    }
  }
}

\end{multicols}
\end{minipage}\\
\end{flushleft}

\noindent In \Last[a], all pointers are bound by their local PDRS, meaning
that all material is accommodated locally.\footnote{We here assume
  a non-specific reading of the indefinite ``a man'', in which it takes
  narrow scope under the negation operator. See
section~\ref{sec:categorization} for a discussion of the specific use of
indefinites.} Note that this representation is identical to the standard DRT
representation of this sentence, except with the addition of labels for
PDRSs and pointers for all referents and conditions.
In \Last[b], on the other hand, the proper name ``John'' triggers
a presupposition about the existence of an entity called `John'.  This
presupposition projects, and is accommodated at the global context, since no
antecedent for binding is available.  While in standard DRT this would be
represented by moving the projected content to the context of
interpretation, in PDRT the content can simply stay at its introduction
site, while projection is signalled via the use of a free pointer, as
indicated by the variable ``f''.  If no further context is added, the free
pointer is simply interpreted at the most global discourse level; here the
PDRS labeled ``1''. The resulting interpretation can be described using the
following first-order logic formula:  $\exists x (John(x) \wedge \neg
\exists y (person(y) \wedge see(y,x)))$.

So, instead of moving semantic material on the representation level to the
context of interpretation, as in \cites{sandt1992presupposition} account,
projection is realized by setting a variable equation between pointers and
context labels.  This allows for a compositional treatment of projection
phenomena that does not assume a two-stage resolution algorithm.  Moreover,
the resulting semantic representation stays close to the linguistic
surface structure, making it easy to evaluate.  Only at the interpretation
stage, i.e., when providing a model-theoretic interpretation of the
representation structures, the content needs to be moved to its
accommodation site in order to obtain the desired interpretation, as we will
see in the PDRS semantics described below.

\subsection{Syntax and semantics}

The basic meaning-carrying units in PDRT are called Projective Discourse
Representation Structures (PDRSs). Each PDRS introduces a new label, and all
referents and conditions in a PDRS are associated with a label via
a pointer. This pointer may be bound by the label of the local PDRS, or by
a label of a PDRS that is accessible from the local PDRS, or it may occur
free. The syntax of a PDRS is recursively defined as follows:

\begin{definition}[PDRS] A PDRS is a triple $\langle l, D, C\rangle$, such that:
  \begin{itemize}
    \item $l$ is a label;
    \item $D$ is a set of projected discourse referents $\langle p,d\rangle$, 
      such that $p$ is a pointer and $d$ is a discourse referent;
    \item $C$ is a set of projected conditions $\langle p, c\rangle$, such 
      that $p$ is a pointer and $c$ is a condition, defined as follows:
      \begin{itemize}
        \item if $P$ is an $n$-place predicate and $u_1,...,u_n$ are discourse
          referents, then $P(u_1,...,u_n)$ is a condition;
        \item if $\phi$ and $\psi$ are PDRSs, then $!\phi$, $\neg \phi$,
          $\phi \vee \psi$ and $\phi \Rightarrow \psi$ are conditions.
      \end{itemize}
  \end{itemize}
\end{definition}

\noindent This syntax closely follows the DRS syntax in standard DRT
\citep{kamp1993discourse}, with the addition of labels for PDRSs and
pointers for the projected referents and conditions of PDRSs. Moreover, an
extra operator $!$ is added, which is interpreted as the opposite of
negation (i.e., assertion), thus indicating positive information
\citep[cf.][]{bos2009controlled}. This operator will be used to distinguish
the projective behaviour of conventional implicatures and specific
indefinites from presupposition projection.

%%%XXX real definition projection path, accessibility.
Binding is possible within the \emph{projection
path} of a PDRS $K_n$ in which some projected content is introduced. The
projection path is defined as in \citet{sandt1992presupposition}: a sequence
of PDRSs $\langle K_0,...,K_n\rangle$, such that each member immediately
subordinates the next one, where subordination is defined in the standard
DRT way \citep[cf.][]{kamp1993discourse}. Additionally, all presupposed
PDRSs (i.e., the PDRSs introduced by free pointers) are on the projection
path of any PDRS. 

The semantics of PDRT can be given via a translation to DRT
\citep[see][]{venhuizen2013pdrt}, or directly to first-order logic.  This
involves a three-step translation in which first the PDRS is traversed,
collecting all pointed referents, pointed conditions and embedded PDRSs
(\emph{accommodation sites}) in separate lists. In the second step, all
pointed referents and conditions are projected to their appropriate
accommodation site; either the PDRS with the label that matches the pointer,
or the global PDRS in case of a free pointer.  When all material is moved to
the correct interpretation site, the final step is to translate the list of
accommodation sites to a standard DRS or a first-order logic formula.
\emph{[see appendix for full translation?]}

This translation results in an interpretation of projection phenomena that
is equivalent to the one proposed by \citet{sandt1992presupposition}. The
only difference is the construction and representation of this
interpretation; where \citeauthor{sandt1992presupposition} uses a two-step
algorithm to construct a representation, in PDRT this construction proceeds
incrementally, and only on the interpretation level the semantic material
needs to be moved to its interpretation site.  Thus, PDRT is simply
a representational framework that allows for a compositional construction of
semantic representations that stay close to the linguistic surface
structure. In terms of the model-theoretic interpretation, however, these
structures are equivalent to standard semantic representations, that do not
contain information about the surface structure. This way, we obtain the
best of both worlds; a representation that can rely on the model-theoretic
properties and inference mechanisms of established semantic representations
(i.e., DRT and first-order logic), enriched with compositional and
structural information about the behaviour of projection phenomena.

\subsection{PDRT in action}

\subsubsection{Presuppositions and anaphora}

As described above, the representation of presuppositions in PDRT follows
\cites{sandt1992presupposition} DRT analysis in terms of treating
presupposition projection in the same way as anaphora resolution. We do,
however, take this approach more literally, since in PDRT projection is
treated as variable binding, just like anaphoric expressions in DRT are
bound by their antecedent. Moreover, the PDRT representations stay closer to
the linguistic surface form than \citeauthor{sandt1992presupposition}'s
resolved DRSs, since projected material is not moved to its interpretation
site. In order to maintain coherence, the same strategy is applied to
anaphora in PDRT; they appear bound by their antecedent at their
introduction site. This is illustrated in \Next.

\begin{flushleft}
\begin{minipage}{0.95\linewidth}
\begin{multicols}{2}
\ex. \a. A boy runs. Nobody sees him.\\\\
\hspace*{-0.4cm}{\small
  \pdrs{1}{$1\gets x$}{
    $1\gets$boy($x$)\\
    $1\gets$run($x$)\\
    $1\gets\neg$\pdrs{2}{$2\gets y, 1\gets x$}{
      $2\gets$person($y$)\\ $2\gets$see($x,y$)
    }
  }
}
\columnbreak
\b. A boy runs. Nobody sees the 15-year-old thief.\\
\hspace*{-0.8cm}
{\small
  \pdrs{1}{$1\gets x$}{
    $1\gets$boy($x$)\\
    $1\gets$run($x$)\\
    $1\gets\neg$\pdrs{2}{$2\gets y, 1\gets x$}{
      $2\gets$person($y$)\\ $2\gets$see($x,y$)\\ $1\gets$15-year-old($x$)\\
      $1\gets$thief($x$)
    }
  }
}

\end{multicols}
\end{minipage}
\end{flushleft}

\noindent In \Last[a], the pronoun ``him'' is bound by the introduction of
``a man'' in the first sentence. Instead of just reusing the discourse
referent introduced by the antecedent as in standard DRT, the same discourse
referent is newly introduced in the local PDRS of the pronoun, with
a pointer to the PDRS of its antecedent.  Formally, unique discourse
referents and conditions are defined over the set of pointers rather than
the set of PDRSs; this allows pointed referents/conditions to be repeated in
the case of (presupposition) binding.  Repeating the introduction of the
discourse referent enhances readability of texts with long-distance
dependencies and reflects the strong correspondence between anaphoric
expressions and presuppositions, as illustrated in \Last[b]. Just like the
pronoun, the definite description in this sentence introduces a bound
discourse referent. Additionally, however, the content of the definite
description is projected to the PDRS of the antecedent. This results in the
desired truth conditions, which state that `there is a boy that is
a 15-year-old-thief that is running and nobody sees him'. On the other hand,
the structural properties of the discourse are clearly reflected in the
representation; the subject is first introduced as a boy, and later in the
discourse it becomes clear that this subject is also a 15-year-old thief.

As we have seen, the referent introduced by the definite description in
\Last[b] occurs bound, but the rest of the presupposed content (that the
subject is a 15-year-old thief) is not bound by the context. In this case,
the non-bound presupposed content is accommodated at the binding site of the
referent. This resolution strategy is also known as \emph{intermediate
accommodation}, since the content accommodates at a level between the local
context (the introduction site) and the global context.
%
%An important aspect of \cites{sandt1992presupposition} theory are the
%predictions it makes with respect to determining how a presupposition is
%resolved. \citet{beaver2002presupposition} proposes some issues with
% \cites{sandt1992presupposition} heuristics for determining projection
% behaviour of presuppositions.
%
%
\subsubsection{Factives}

\subsubsection{Conventional implicatures}

Conventional implicatures, as described by \citet{potts2005logic}, have
received much less attention in terms of semantic representation than
presuppositions. However, much work has been done on the syntactical and
combinatorial behaviour of CIs, in particular that of appositive
constructions \citep[see, e.g.][]{mccawley1998syntactic,
delgobbo2003appositives,nouwen2007appositives,schlenker2010supplementsi,
heringa2012appositional}.  This discussion mainly evolves around the
relation between the appositive, and the \emph{anchor} to which it attaches.
For example, the anchor in \Next is ``John'' and the (nominal) appositive
that connects to this anchor is the clause ``a professor''.

\ex. John, a professor, lives in Groningen.

This construction triggers the CI ``John is a professor''. In order to
obtain this reading, \citet{potts2005logic} treats the appositive as
a predicate for the anchor in his type-theoretical system.  However, as
\citet{nouwen2013note} notes, the behaviour of appositives can be explained
more parsimoniously by treating appositives as propositions containing
a pronoun that is (discourse) anaphoric to its anchor
\citep[following][]{delgobbo2003appositives}. This means that the sentence
in \Last is logically equivalent to~\Next: 

\ex. John$_i$ lives in Groningen. He$_i$ is a professor.

In this line, \citet{schlenker2013supplements} proposes a unidimensional
account of supplemental expressions, including appositives and
non-restrictive relative clauses (NRRCs).  He states that supplements are
semantically discourse referential to the rest of the sentence, but they are
subject to a pragmatic rule that requires that their content be ``easy to
accommodate'' and ``non-trivial''. 
Here, we can interpret non-triviality as
not allowing resolution by means of binding, since by definition some
content is trivial if it is already established in the current context, and
hence can act as an antecedent for binding.

In PDRT, we can formalize this behaviour by adding the CI content, embedded
by the $!$~operator, to the anchor and projecting them together to the
appropriate discourse level.  The exclamation mark simply indicates
asserted content, but because of the embedding it prevents the pointers of
the content from becoming bound by higher context labels.  The anchor of
a CI construction determines the projection site of the complete CI, and
while the anchor and presuppositional parts of the CI can bind to available
antecedents, the CI itself remains accommodated within the context created
by the exclamation mark. Example \Next shows an example of an indefinite
nominal appositive, and a definite nominal appositive, which is
a presupposition trigger itself.

\begin{flushleft}
\begin{minipage}{\linewidth}
\begin{multicols}{2}
  \ex. \a.\hspace*{-0.2cm} John, a linguist, is coming.\\
\hspace*{-0.5cm}{\small \pdrs{1}{$f\gets x$}{$f\gets$John($x$)\\ 
    $f\gets~!$\pdrs{2}{$2\gets y$}{$2\gets$linguist($y$)\\ $2\gets y=x$}\\ 
$1\gets$coming($x$)}
}
\columnbreak
\b.\hspace*{-0.2cm} John, the linguist, is coming.\\
\hspace*{-0.5cm}{\small \pdrs{1}{$f_1\gets x$}{$f_1\gets$John($x$)\\ 
    $f_1\gets~!$\pdrs{2}{$f_2\gets y$}{$f_2\gets$linguist($y$)\\ $2\gets y=x$}\\ 
    $1\gets$coming($x$)}
}

\end{multicols}
\end{minipage}
\end{flushleft}

[explanation]

\subsection{Constructing a discourse representation}\label{sec:composition}

The construction of a PDRT representation of a discourse proceeds
incrementally, by combining the semantics of lexical items to form
a representation of the entire discourse. We apply a variant of
compositional DRT, as proposed by \citet{muskens1996combining}, which
associates lexical items with an unresolved semantics in the form of
a lambda term \citep[see][for a computational
implementation]{bos2003implementing}. These lambda terms determine how PDRSs
are combined using \emph{merge-reduction}. In PDRT, we define different
types of \emph{merge} to reflect the different ways in which new material
can be integrated into the discourse.

\subsubsection{Merge operations}

The standard merge in compositional DRT is defined as the
union of the referents and the conditions of the two merged DRSs. However,
the addition of pointers in PDRT complicates this procedure, since there are
different ways in which the pointers can behave, indicating different types
of projection behaviour.  In the case of asserted content, the PDRS
resulting from the merge should have its referents and conditions locally,
i.e., with pointers to the label of the local PDRS. So, when combining some
asserted content $A$ with a context $B$, the \emph{assertive merge} renames
the (local) pointers of $A$ to the label of $B$, resulting in a PDRS where
the local content of PDRS $A$ becomes locally accommodated in PDRS $B$. The
definition of assertive merge is shown below. For the renaming of pointers
we use the notation `$A[x/y]$', which is taken to represent the set
resulting from replacing every instance of $y$ in the set $A$ by $x$.

\begin{definition}[Assertive merge]
{\small \pdrs{$i$}{$D_i$}{$C_i$} $+$ 
 \pdrs{$j$}{$D_j$}{$C_j$}  \ := \ 
 \pdrs{$j$}{$D_i[j/i]\cup D_j$}{
    $C_i[j/i]\cup C_j$}}
\end{definition}

\noindent Note that this operation is commutative, since the order of
combining PDRSs does not matter; changing the order of the operands only
results in a PDRS with a different label ($i$, in this case), but the
interpretation stays the same because the locally accommodated content of
both operands stays local. In general, however, we will assume that the
first operand reflects the novel content that needs to be integrated into
the context represented in the second operand.  Assertive merge is also an
associative operation, meaning that multiple instances of assertive merge can
be performed in any order (i.e., for all PDRSs $A$, $B$ and $C$: $A+(B+C)
\equiv (A+B)+C$).

An example of the application of assertive merge is shown in \Next.

\ex. (The man smiles) $+$ (a woman cries)\\
{\small
\pdrs{1}{$f\gets x$}{$f\gets$man($x$)\\ $1\gets$smile($x$)} 
  $+$ \pdrs{2}{$2\gets y$}{$2\gets$woman($y$)\\ $2\gets$cry($y$)}
  := \pdrs{2}{$f\gets x, 2\gets y$}{$f\gets$man($x$)\\ $2\gets$smile($x$)\\ 
        $2\gets$woman($y$)\\ $2\gets$cry($y$)}
}

After using assertive merge, all projected content in the operands remains
projected content in the resulting PDRS, and the asserted content of each of
the operands remains locally accommodated with a pointer to the local
context, as desired.

As described above, the difference between assertions and presuppositions in
PDRT is that assertions obtain as pointer the label of their local PDRS,
while presuppositions can project to any PDRS on the projection path. This
means that, in the spirit of \citet{sandt1992presupposition}, the antecedent
of the pointer of a presupposition is determined in the same way as the
antecedent of an anaphoric expression; based on a set of heuristics that
selects the appropriate antecedent from a list of possible antecedents.
Provided that the correct accommodation site of a presupposition has been
determined in this way, \emph{projective merge} can simply be defined as
adding the referents and conditions of the presupposed PDRS to the rest of
the context, without changing the pointers of the presupposed content. This
way, presuppositions are added to their local context, while maintaining the
accommodation site that has been determined by predefined heuristics. In
case the accommodation site of the presupposition is not available in the
final representation, the pointer remains free and the presupposition is
interpreted at the global discourse level. 
%This results in the desired behaviour for presuppositions; in case some
%presupposed content is not constituted in the current discourse context,
%the content should be added to the common ground, which is here represented
%as the global context.  
The definition of projective merge is shown below:

\begin{definition}[Projective merge] 
  {\small ~\pdrs{$i$}{$D_i$}{$C_i$} $\ast$ 
 \pdrs{$j$}{$D_j$}{$C_j$}  \ := \ 
 \pdrs{$j$}{$D_i\cup D_j$}{$C_i\cup C_j$}}
\end{definition}

\noindent Projective merge is associative, since it holds that multiple
assertive merge operations can be performed in any order. This only holds as
long as the order of the operands is not changed, since contrary to
assertive merge, projective merge is not commutative. The order of the
operands crucially determines which content gets projected; by labeling the
resulting PDRS with the label of the second operand, the local content of
the first operand becomes projected.  That is, in case two distinctly
labeled PDRSs are merged (i.e., $i\neq j$), the local content in the first
operand becomes free in the final PDRS.  Note that once some content is
projected (associated with a free label), it will stay projected in every
subsequent step of the composition, except when it gets bound. The only way
for projected content to get bound is if it is merged (either by projective
of assertive merge) into a PDRS whose label corresponds to the pointer of
the projected content. 

%Because of the different properties of projective and assertive merge, their
%interaction is non-trivial, as illustrated in \Next. 
%
%\ex.  For some PDRSs $A, B$ and~$C$ with different labels, the following holds:
% \a. $A * (B + C) = (A*B) + C$ 
% \b. $A + (B * C) \neq (A+B) * C$

\subsubsection{Lexical semantics}

%Using projective and assertive merge we can define a lexical semantics in
%the form of an unresolved lambda-term for all lexical items.  We assume
%a Montagovian compositional semantics for PDRT, based on
%\citeauthor{muskens1996combining}' compositional DRT
%\citet{muskens1996combining}. The lexical representations provided here are
%similar to those proposed by \citet{bos2003implementing}.

%Example \Next shows the lexical entries for the definite article
%``the'' and the indefinite article ``a''. While the definite article signals
%that its complement is presupposed, i.e., refers to a (unique) salient
%referent available in the current context or in the common ground, the
%indefinite article merely signals that a new entity is introduced in the
%local context. 

%[more explanation, examples, derivations]

\begin{tabular}{l | c | c}
{\bf Lexical item} & {\bf Category} & {\bf Semantics}\\
\hline
%a & NP/N & $\lambda p.\lambda q.(($\small{\pdrs{$1$}{$1\gets x$}{}}$+~p(x))~+~q(x) )$\\
%the & NP/N & $\lambda p.\lambda q.(($\small{\pdrs{$1$}{$1\gets x$}{}}$+~p(x))~*~q(x) )$\\
%John & NP &  $\lambda p.($\small{\pdrs{$1$}{$1\gets x$}{$1\gets$John($x$)}}$~*~p(x) )$\\
%man & N &  $\lambda x.$\small{\pdrs{$1$}{}{$1\gets$man($x$)}}\\
%walks & S$\backslash$NP & $\lambda p. \lambda q. (p(\lambda x. ($\pdrs{1}{
%  $1\gets e$}{$1\gets$walk($e$)\\ $1\gets$Theme($e,x$)}$ +~q(e))))$\\
knows & (S$\backslash$NP)/NP & $\lambda p. \lambda q. \lambda r.(q(\lambda x.($\pdrs{1}{
$1\gets e$ $1\gets p1$}{$1\gets$know($e$)\\ $1\gets$Agent($e,x$)\\ $1\gets$Theme($e,p$)\\ $1\gets$p1:$(p(\lambda y.$\pdrs{2}{}{}$) \ast p(\lambda z.$\pdrs{3}{}{}))}$ +~r(e))))$\\
believes & (S$\backslash$NP)/NP & $\lambda p. \lambda q. \lambda r.(q(\lambda x.($\pdrs{1}{
$1\gets e$ $1\gets p1$}{$1\gets$believe($e$)\\ $1\gets$Agent($e,x$)\\ $1\gets$Theme($e,p$)\\ $1\gets$p1:$(p(\lambda y.$\pdrs{2}{}{}$) + p(\lambda z.$\pdrs{3}{}{}))}$ +~r(e))))$\\

\end{tabular}\\



