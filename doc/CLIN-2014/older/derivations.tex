Figure~\ref{fig:deriv_definite} shows an example derivation of a sentence
containing a definite description.
%%%XXX explain

\begin{figure}[h!t]
\centering
\begin{footnotesize}
  \begin{tabular}{c c c}
    The: NP/N & man: N\\
$\lambda p.\lambda q.(($\pdrs{$1$}{$1\gets x$}{}$+~p(x))*q(x))$ &
    $\lambda x.($\pdrs{2}{}{$2\gets$man($x$)})\\
\cline{1-2}\\
\multicolumn{2}{l}{\hspace{3cm}The man: NP}\\
\multicolumn{2}{l}{\hspace{1cm}$\lambda q.(($\pdrs{$1$}{$1\gets x$}{}$+
\pdrs{2}{}{$2\gets$man($x$)})\ast q(x))$}\\
\cline{1-2}\\
The man: NP & \multicolumn{2}{c}{walks: S$\backslash$NP}\\
$\lambda q.($\pdrs{$2$}{$2\gets x$}{$2\gets$man($x$)}$ \ast~q(x))$ &
\multicolumn{2}{r}{$\lambda p. \lambda q. (p(\lambda x. ($\pdrs{3}{
  $3\gets e$}{$3\gets$walk($e$)\\ $3\gets$Theme($e,x$)}$ +~q(e))))$}\\
\hline\\
\multicolumn{3}{c}{The man walks: S}\\
\multicolumn{3}{c}{
$\lambda q. (($\pdrs{$2$}{$2\gets x$}{$2\gets$man($x$)}$\ast$
  \pdrs{3}{$3\gets e$}{$3\gets$walk($e$)\\ $3\gets$Theme($e,x$)}$) + q(e))))$}\\
\hline\\
\multicolumn{3}{c}{The man walks: S}\\
\multicolumn{3}{c}{$\lambda q. ($\pdrs{3}{$2\gets x, 3\gets e$}{
  $2\gets$man($x$)\\$3\gets$walk($e$)\\ $3\gets$Theme($e,x$)}$+~q(e))))$}\\
\end{tabular}
\end{footnotesize}
\caption{Derivation of the sentence: ``The man walks''.}
\label{fig:deriv_definite}
\end{figure}

Figure~\ref{fig:deriv_ci} shows an example derivation of a sentence
containing an indefinite appositive.

\begin{sidewaysfigure}[b!]
\begin{footnotesize}
  \begin{tabular}{p{2cm} c c c}
  & \multicolumn{2}{c}{APPCOMMA : (N$\backslash$N)/NP} 
    & a linguist: NP\\
  & \multicolumn{2}{r}{$\lambda p. \lambda q. \lambda x. ($
      \pdrs{2}{}{$2\gets~!~p(\lambda y. ($\pdrs{3}{}{$3\gets y=x$}$))$}$+~q(x))$}
    & $\lambda q.($\pdrs{$4$}{$4\gets y$}{$4\gets$linguist($y$)}$ +~q(y))$\\
  \cline{2-4}\\
%  & \multicolumn{3}{r}{APPCOMMA a linguist: N$\backslash$N\hspace*{4.5cm}}\\
%  & \multicolumn{3}{r}{$\lambda q. \lambda x. ($\pdrs{2}{}{$2\gets~!$(\pdrs{4}{
%      $4\gets y$}{$4\gets$linguist($y$)}$+$\pdrs{3}{}{$3\gets y=x$})}$
%      +~q(x))$\hspace*{2cm}}\\
%  \cline{2-4}\\
  \multicolumn{1}{c}{John: N} 
    & \multicolumn{2}{c}{APPCOMMA a linguist: N$\backslash$N}\\
  \multicolumn{1}{c}{$\lambda x.($\pdrs{1}{}{
    $1\gets$John($x$)})}
    & \multicolumn{2}{r}{$\lambda q. \lambda x. ($\pdrs{2}{}{
      $2\gets~!$\pdrs{3}{$3\gets y$}{$3\gets$linguist($y$)\\ $3\gets y=x$}}$
      +~q(x))$}\\
  \cline{1-3}\\
  \multicolumn{3}{c}{John APPCOMMA a linguist: N}\\
  \multicolumn{3}{c}{$\lambda x. ($\pdrs{2}{}{$2\gets$John($x$)\\
      $2\gets~!$\pdrs{3}{$3\gets y$}{$3\gets$linguist($y$)\\ $3\gets y=x$}})}\\
  \cline{1-3}\\
  \multicolumn{2}{c}{John APPCOMMA a linguist: NP} 
    & \multicolumn{2}{r}{walks: S$\backslash$NP\hspace*{2.5cm}}\\
  \multicolumn{2}{c}{\hspace{0.5cm}$\lambda p. ($\pdrs{2}{$2\gets x$}{
      $2\gets$John($x$)\\ $2\gets~!$\pdrs{3}{$3\gets y$}{$3\gets$linguist($y$)\\ 
      $3\gets y=x$}}$\ast~p(x))$}
    & \multicolumn{2}{r}{$\lambda p. \lambda q. (p(\lambda x. ($\pdrs{5}{
      $5\gets e$}{$5\gets$walk($e$)\\ $5\gets$Theme($e,x$)}$ +~q(e))))$}\\
  \hline\\
%  \multicolumn{4}{c}{John APPCOMMA a linguist walks: S}\\
%  \multicolumn{4}{c}{$\lambda q.($\pdrs{2}{}{$2\gets$John($x$)\\
%      $2\gets~!$\pdrs{3}{$3\gets y$}{$3\gets$linguist($y$)\\ $3\gets y=x$}}$\ast$
%    \pdrs{5}{$5\gets e$}{$5\gets$walk($e$)\\ $5\gets$Theme($e,x$)}$)+q(e))$}\\
%  \hline\\
%  We here use the equivalence: A*(B+C) =:= (A*B)+C
  \multicolumn{4}{c}{John APPCOMMA a linguist walks: S}\\
  \multicolumn{4}{c}{$\lambda q. ($\pdrs{5}{$2\gets x,5\gets e$}{
      $2\gets$John($x$)\\ $2\gets~!$\pdrs{3}{$3\gets y$}{$3\gets$linguist($y$)\\ 
      $3\gets y=x$}\\ $5\gets$walk($e$)\\ $5\gets$Theme($e,x$)}$+ q(e))$}
 \end{tabular}
\end{footnotesize}
\caption{Derivation of the sentence: ``John,$_{app}$ a linguist, walks.''.}
\label{fig:deriv_ci}
\end{sidewaysfigure}

%%%%%%%%%%%%%%%%%%%%%%%%%%%%%%%%%%%%COMMENT%%%%%%%%%%%%%%%%%%%%%%%%%%%%%%%%%%%%
\begin{comment}
%% old lexical definition for appcomma using implicative merge
\begin{sidewaysfigure}[b!]
\begin{footnotesize}
  \begin{tabular}{p{2cm} c c c}
  & \multicolumn{2}{c}{APPCOMMA : (NP$\backslash$NP)/NP} 
    & a linguist: NP\\
  & \multicolumn{2}{r}{$\lambda p. \lambda q. \lambda r. (q(\lambda x. ((p(\lambda y.
    ($\pdrs{2}{}{$2\gets y = x$}$))) \bullet r(x))))$}
    & $\lambda q.($\pdrs{$3$}{$3\gets y$}{$3\gets$linguist($y$)}$ +~q(y))$\\
  \cline{2-4}\\
  & \multicolumn{3}{r}{APPCOMMA a linguist: NP$\backslash$NP\hspace*{4.5cm}}\\
  & \multicolumn{3}{r}{$\lambda q. \lambda r. (q(\lambda x. (($\pdrs{$3$}{
    $3\gets y$}{$3\gets$linguist($y$)}$+$\pdrs{2}{}{$2\gets y = x$}$) 
    \bullet r(x))))$\hspace*{2cm}}\\
  \cline{2-4}\\
  \multicolumn{1}{c}{John: NP} 
    & \multicolumn{2}{c}{APPCOMMA a linguist: NP$\backslash$NP}\\
  \multicolumn{1}{c}{$\lambda p.($\pdrs{1}{$1\gets x$}{
    $1\gets$John($x$)}$\ast~p(x))$}
    & \multicolumn{2}{c}{$\lambda q. \lambda r. (q(\lambda x. (($\pdrs{2}{
      $2\gets y$}{$2\gets$linguist($y$)\\ $2\gets y = x$}$) \bullet r(x))))$}\\
  \cline{1-3}\\
  % Adding rule for COMMA is uninteresting and unnecessary, since it is
  % semantically empty.
%  \multicolumn{2}{c}{John APPCOMMA a linguist: NP} 
%    & COMMA: NP$\backslash$NP\\
%  \multicolumn{2}{c}{$\lambda r. ($\pdrs{1}{$1\gets x$}{$1\gets$John($x$)}$
%    \ast~($\pdrs{2}{$2\gets y$}{$2\gets$linguist($y$)\\ $2\gets y = x$}$ 
%    \bullet~r(x)))$} 
%    & $\lambda p. p$\\
%  \cline{1-3}\\
  \multicolumn{2}{c}{John APPCOMMA a linguist: NP} 
    & \multicolumn{2}{r}{walks: S$\backslash$NP\hspace*{2.5cm}}\\
  \multicolumn{2}{c}{\hspace{1cm}$\lambda r. ($\pdrs{1}{$1\gets x$}{$1\gets$John($x$)}$
    \ast~($\pdrs{2}{$2\gets y$}{$2\gets$linguist($y$)\\ $2\gets y = x$}$ 
    \bullet~r(x)))$}
    & \multicolumn{2}{r}{$\lambda p. \lambda q. (p(\lambda x. ($\pdrs{3}{
      $3\gets e$}{$3\gets$walk($e$)\\ $3\gets$Theme($e,x$)}$ +~q(e))))$}\\
  \hline\\
  \multicolumn{4}{c}{John APPCOMMA a linguist walks: S}\\
  \multicolumn{4}{c}{$\lambda q.(($\pdrs{1}{$1\gets x$}{$1\gets$John($x$)}$
    \ast~($\pdrs{2}{$2\gets y$
    }{$2\gets$linguist($y$)\\ $2\gets y = x$}$\bullet$
    \pdrs{3}{$3\gets e$}{$3\gets$walk($e$)\\ $3\gets$Theme($e,x$)}$))+q(e))$}\\
  \hline\\
  % The following step may be skipped for reasons of space. However, it
  % makes the need for a projecting expression as anchor very explicit; if
  % in this step an assertive rather than projective merge is used, the
  % variable in the implied equation statement (x) becomes free.
  \multicolumn{4}{c}{John APPCOMMA a linguist walks: S}\\
  \multicolumn{4}{c}{$\lambda q.(($\pdrs{1}{$1\gets x$}{$1\gets$John($x$)}$
    \ast~$\pdrs{3}{$0\gets y, 3\gets e$}{$0\gets$linguist($y$)\\ 
    $0\gets y = x$\\ $3\gets$walk($e$)\\ $3\gets$Theme($e,x$)}$)+q(e))$}\\
  \hline\\
  \multicolumn{4}{c}{John APPCOMMA a linguist walks: S}\\
  \multicolumn{4}{c}{$\lambda q. (\pdrs{3}{$1\gets x, 0\gets y,3\gets e$}{
    $1\gets$John($x$)\\ $0\gets$linguist($y$)\\ $0\gets y = x$\\ 
    $3\gets$walk($e$)\\ $3\gets$Theme($e,x$)} + q(e))$}
 \end{tabular}
\end{footnotesize}
\caption{Derivation of the sentence: ``John,$_{app}$ a linguist, walks.''.}
\label{fig:deriv_ci}
\end{sidewaysfigure}
\end{comment}
%%%%%%%%%%%%%%%%%%%%%%%%%%%%%%%%%%%%COMMENT%%%%%%%%%%%%%%%%%%%%%%%%%%%%%%%%%%%%

