\section{Operations on PDRSs}\label{sec:operations}

\subsection{Properties}



\subsection{Translations}

A projection table (\textsc{ptable}) is an $n\times3$ matrix, which can be
represented as a set of triples, where each triple consists of some PDRS
content (a referent, a predicate or an operator with PDRS label(s)), the
introduction site of the content, and the projection site of the content,
i.e.: \textsc{ptable}= $\{[C,I,P]~|~C=$ PDRS content $\wedge$ $I=$ an
introduction site $\wedge$ $P=$ a projection site$\}$. The translation from
a PDRS to a projection table can be straightforwardly defined using two
operations $\mathcal{T}$ and $\mathcal{I}$, defined in the following seven
clauses (here, $O^1$ indicates a unary operator: i.e., one of $\{\neg, \Box,
\Diamond$\}; and $O^2$ a binary operator: i.e., one of $\{\Rightarrow,
\vee\}$).

\begin{definition}[PDRS to Projection Table]~\\\vspace{-12pt}
\begin{enumerate}
  \item $\mathcal{T}(\langle\rho,\{\mu_1...\mu_n\},
    \{\delta_1...\delta_m\}, \{\chi_1...\chi_l\}\rangle$)\\
    = $\mathcal{I}(\mu_1,\rho)\cup...\cup \mathcal{I}(\mu_n,\rho)\cup
      \mathcal{I}(\delta_1,\rho)\cup...\cup \mathcal{I}(\delta_m,\rho)\cup
      \mathcal{I}(\chi_1,\rho)\cup...\cup \mathcal{I}(\chi_l,\rho)$
      %What about the MAPs?
  \item $\mathcal{I}(\langle v_1, v_2 \rangle,\rho_i)
    = \{[\langle v_1, v_2 \rangle,\rho_i,\rho_i]\}$
  \item $\mathcal{I}(\langle v_i, x_i \rangle,\rho_i)
    = \{[x_i,\rho_i,v_i]\}$
  \item $\mathcal{I}(\langle v_i, R(x_1...x_j) \rangle,\rho_i)
    = \{[R(x_1...x_j),\rho_i,v_i]\}$
  \item $\mathcal{I}(\langle v_i, O^1 K_j) \rangle,\rho_i)
    = \{[O^1 \rho(K_j),\rho_i,v_i]\} \cup \mathcal{T}(K_j)$
  \item $\mathcal{I}(\langle v_i,  O^2(K_j, K_k) \rangle,\rho_i)
    = \{[O^2(\rho(K_j),\rho(K_k)),\rho_i,v_i]\} \cup \mathcal{T}(K_j)
      \cup \mathcal{T}(K_k)$
  \item $\mathcal{I}(\langle v_i, x_i:K_j \rangle,\rho_i)
    = \{[x_i:\rho(K_j),\rho_i,v_i]\} \cup \mathcal{T}(K_j)$
\end{enumerate}
\end{definition}

To reconstruct a PDRS from a projection table, we need to
retrieve the information about the structure of the original PDRS,
and then insert all other PDRS content at the correct introduction site. For
the first step, we need to know the label of the outer PDRS, and the labels
of all sub-PDRSs. The latter information can be simply derived from the PDRS
content represented in \textsc{ptable} $T$: \textsc{sub-pdrss} $=\{l~|~O^1
  l \in \bigcup C ~\vee~ O^2(l,l') \in \bigcup C ~\vee~ O^2(l',l) \in
\bigcup C ~\vee~ x:l \in \bigcup C \}$, where $\bigcup C$ represents all
PDRS content represented in \textsc{ptable} $T$. Since the label of the
outer PDRS by definition is the only label that occurs as an introduction
site but not as a sub-PDRS, it can be retrieved as follows:
\textsc{outer-label} $=\{i~|~i\in\bigcup I \wedge i \notin
\textsc{sub-pdrss}\}$, where $\bigcup I$ is the collection of all
introduction sites in $T$.

With this information, we can re-create the original PDRS starting from an
empty PDRS with the \textsc{outer-label} as label. Then all other content
can be incrementally inserted on the basis of the introduction site
associated with the PDRS content. When translating a PDRS to a DRS, all
content needs to be moved to the projection site, instead of the
introduction site, since the information provided by the pointers will be
lost, so all content needs to occur at its interpretation site in order to
obtain the desired truth-conditional interpretation. One constraint on this
operation is the occurrence of free pointers the projection site of
presupposed content. These free contexts can be retrieved from the
projection table by collecting all projection contexts that do not occur
within the \textsc{sub-pdrss} or as the \textsc{outer-label} of $T$:
\textsc{free-contexts} $=\{p~|~p\in\bigcup P \wedge p \notin
\textsc{sub-pdrss} \wedge p \notin \textsc{outer-label}\}$, where $\bigcup
P$ is the collection of all projection sites in $T$.
%what about an empty PDRS? With this def of projection table it is
%impossible to retrieve its label.. Idea: Maybe add recursive accessibility
%of label in the same way as maps? Yes!

In order to determine the interpretation site of such presupposed content,
the \textsc{free-contexts} need to be translated into the labels from the
\textsc{sub-pdrss} or the \textsc{outer-label}. For this, we can use the
\MAPs, which are available in the projection table as empty PDRS content.

\subsection{Higher-order functions}
%pdrsIsSameNP etc
