\section{Building Projective Discourse Representation Structures}

% nice introduction

\subsection{Basic Structures}

We define basic DRSs following \citeasnoun{bos2003implementing}, with the
exception that equality between variables (i.e., $x_1=x_2$) is not
a separate DRS condition, but is treated as a variant of a 2-place predicate
(i.e., $R(x_1,x_2)$, where $R$ equals $=$). This definition basically
follows the definition proposed by \citeasnoun{kamp1993discourse}, without
their duplex conditions, in order to allow for a direct translation to
first-order logic.

\begin{definition}[Basic DRS] \label{def:bDRS}
 A Basic DRS is a tuple $\langle \{x_1 ... x_n\},\{\gamma_1 ... \gamma_m\} 
 \rangle$, where:
 \begin{enumerate}[i]
  \item $\{x_1 ... x_n\}$ is a finite set of variables (called ``DRS
    referents'');
  \item $\{\gamma_1 ... \gamma_m\}$ is a finite set of DRS conditions (which
    may be either basic or complex);
  \item\label{def:bDRS:Rel} $R(x_1, ..., x_n)$ is a basic DRS condition,
    with $x_1 ... x_n$ are variables and $R$ is a relation symbol for an
    $n$-place predicate;
  \item $\neg K$, $\Box K$ and $\Diamond K$ are complex DRS conditions, with
    $K$ is a DRS;
  \item $K_1 \vee K_2$ and $K_1 \Rightarrow K_2$ are complex DRS conditions,
    with $K_1$ and $K_2$ are DRSs;
  \item \label{def:bDRS:Prop} $x:K$ is a complex DRS condition, with $x$ is
    a variable and $K$ is a DRS.
 \end{enumerate} 
\end{definition}

%%more explanation?

The main difference between a basic DRS and a basic PDRS is that in PDRT all
structures introduce a label, which can accordingly bind the pointers
associated with the PDRS referents and conditions. We will refer to labels
and pointers together as \textit{projection variables}. This description
results in the definition of a PDRS as a triple of a projection variable,
a set of projected referents (i.e., DRS referents associated with a pointer)
and a set of projected conditions. However, such a definition omits an
essential part of the information that is required to properly represent
projection phenomena, namely their \emph{projection path}. In DRT, the
projection path (or ``projection line'', as
\citeasnoun{sandt1992presupposition} calls it) of some projected content
determines which DRSs are accessible from the introduction site to find
a possible antecedent in or otherwise to accommodate into. Accessibility
between DRSs is standardly defined based on a subordination relation
($\leq$) between DRSs, i.e., DRS $K_1$ is accessible from DRS $K_2$
\textit{iff} $K_2\leq K_1$, where subordination is defined as follows:

\begin{definition}[DRS Subordination]
DRS $K_1$ subordinates DRS $K_2$ ($K_2 \leq K_1$) iff:
  \begin{itemize}
    \item $K_1 = K_2$;
    \item $K_1$ directly subordinates $K_2$ ($K_2 <!~K_1$);
    \item There is a $K_{1'}$, such that $K_1$ subordinates $K_{1'}$ and 
      $K_{1'}$ subordinates $K_2$.
  \end{itemize}
  where: A DRS $K_i$ directly subordinates a DRS $K_j$ ($K_j <!~K_i$) iff:
  \begin{itemize}
    \item $\neg K_j$, $\Box K_j$ or $\Diamond K_j$ is a DRS condition in
      $K_i$;
    \item $x:K_j$ is a DRS condition in $K_i$ for some $x$;
    \item $K_j \Rightarrow K_k$ or $K_j \vee K_k$ is a condition in $K_i$
      for some $K_k$;
    \item There is a $K_k$, such that $K_i \Rightarrow K_j$ or $K_i \vee K_j$
      is a condition in $K_k$.
  \end{itemize}
\end{definition}

In PDRT, projection pointers refer to contexts that may or may not be part
of the current discourse structure. They are, however, related to the PDRS
in which they are introduced via a subordination relation, since the
interpretation site of some linguistic content is necessarily accessible
from its introduction site. Moreover, projected contexts may subordinate
other projected contexts, which happens for example in the case of embedded
presuppositions. So, in order to properly define the projection path of
projected content in PDRT, the subordination properties of projected
contexts should be explicitly part of the PDRS. We do this by enhancing each
PDRS with a set of MAPs: Minimally Accessible Projected contexts, which are
basically tuples describing subordination pairs of contexts. This results in
the following definition of basic PDRSs:

\begin{definition}[Basic PDRS] \label{def:bPDRS}
A Basic PDRS is a quadruple $\langle \rho, \{\delta_1 ... \delta_n\},
  \{\chi_1 ... \chi_m\}, \{\mu_1 ... \mu_l\}\rangle$, where:
  \begin{enumerate}[i]
    \item $\rho$ is a projection variable, called the \textit{label} of the
      PDRS;
    \item $\{\delta_1 ... \delta_n\}$ is a finite set of projected
      referents, with $\delta_i=\langle p_i, x_i\rangle$, such that $p_i$ is
      a projection variable, and $x_i$ is a DRS referent;
    \item $\{\chi_1 ... \chi_m\}$ is a finite set of projected conditions,
      with $\chi_j = \langle p_j,\gamma_j\rangle$, such that $p_j$ is a
      projection variable, and $\gamma_j$ is a PDRS condition (which may be
      either basic or complex);
    \item \label{def:bPDRS:Rel} $R(x_1, ..., x_n)$ is a basic PDRS condition,
      with $x_1 ... x_n$ are variables and $R$ is a relation symbol for an
      $n$-place predicate;
    \item $\neg K$, $\Box K$ and $\Diamond K$ are complex PDRS conditions,
      with $K$ is a PDRS;
    \item $K_1 \vee K_2$ and $K_1 \Rightarrow K_2$ are complex PDRS
      conditions, with $K_1$ and $K_2$ are PDRSs;
    \item\label{def:bPDRS:Prop} $x:K$ is a complex PDRS condition, with $x$
      is a variable and $K$ is a PDRS.
    \item $\{\mu_1 ... \mu_l\}$ is a finite set of MAPs, with $\mu_k=\langle
      p_1,p_2\rangle$, such that $p_1$ and $p_2$ are projection variables,
      and $p_2\leq p_1$.

  \end{enumerate}
\end{definition}

\noindent Note the direct correspondence between Definition
\ref{def:bPDRS}(\ref{def:bPDRS:Rel}-\ref{def:bPDRS:Prop}) and Definition
\ref{def:bDRS}(\ref{def:bDRS:Rel}-\ref{def:bDRS:Prop}); the only difference
between these sets of clauses is that complex PDRS conditions contain
subordinated PDRSs, and complex DRS conditions contain subordinated DRSs.


\subsection{Combining Structures}

\subsubsection{Merges}

\subsubsection{Variable binding}

\subsection{Unresolved Structures}
